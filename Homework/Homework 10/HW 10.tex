\documentclass[12pt]{article} 
\usepackage[utf8]{inputenc}
\usepackage{geometry}
\geometry{letterpaper}
\usepackage{graphicx} 
\usepackage{parskip}
\usepackage{booktabs}
\usepackage{array} 
\usepackage{paralist} 
\usepackage{verbatim}
\usepackage{subfig}
\usepackage{fancyhdr}
\usepackage{sectsty}

\pagestyle{fancy}
\renewcommand{\headrulewidth}{0pt} 
\lhead{}\chead{}\rhead{}
\lfoot{}\cfoot{\thepage}\rfoot{}


%%% ToC (table of contents) APPEARANCE
\usepackage[nottoc,notlof,notlot]{tocbibind} 
\usepackage[titles,subfigure]{tocloft}
\renewcommand{\cftsecfont}{\rmfamily\mdseries\upshape}
\renewcommand{\cftsecpagefont}{\rmfamily\mdseries\upshape} %

\usepackage{amsmath}
\usepackage{amssymb}
\usepackage{empheq}
\usepackage{xcolor}
\renewcommand{\L}[1]{\mathcal{L}\{#1\}}
\newcommand{\ans}[1]{\boxed{\text{#1}}}
\newcommand{\vecs}[1]{\langle #1\rangle}
\renewcommand{\hat}[1]{\widehat{#1}}
\newcommand{\F}[1]{\mathcal{F}(#1)}
\renewcommand{\P}{\mathbb{P}}
\newcommand{\R}{\mathbb{R}}
\newcommand{\qed}{\quad \blacksquare}
\newcommand{\brak}[1]{\langle #1 \rangle}

\title{APMA 0360: Homework 10}
\author{Milan Capoor}
\date{28 April 2023}

\begin{document}
\maketitle
\section*{Problem 1:} In 1 dimensions, suppose the Lagrangian
$L = L(p, z, x)$ is given. Mimic the proof given in lecture to show that the Euler-Lagrange equation associated to the min problem
\[\min I[f] = \int_a^b L(f', f, x)\; dx\]
is 
\[-(L_p(f', f, x))_x + L_z(f', f, x) = 0\]

\color{blue}
Let $g$ be an arbitrary function with $g(a) = g(b) = 0$. Consider 
\[h(t) = I[f + tg] = \int_a^b L(f' + tg', f + tg, x)\; dx \] 
Note that $h(0) = I[f]$ so $h(0)$ is a minimum which implies that $h'(0) = 0$. 

Then, 
\begin{align*}
    h'(t) &= \frac{d}{dt}\left[\int_a^b L(f' + tg', f + tg, x)\; dx\right]\\
    &= \int_a^b \frac{d}{dt} L(f' + tg', f + tg, x)\; dx\\
    &= \int_a^b L_p(f' + tg', f + tg, x)(\frac{d}{dt}(f' + tg'))\\
    &\qquad \qquad +  L_z(f' + tg', f + tg, x)(\frac{d}{dt}(f + tg)) \\
    &\qquad \qquad + L_x(f' + tg', f + tg, x)(\frac{d}{dt}(x))\; dx\\
    &= \int_a^b g'L_p(f' + tg', f + tg, x) + gL_z(f' + tg', f + tg, x)\; dx\\
    &= [gL_p(f' + tg', f + tg, x)]_a^b - \int_a^b g(L_p(f' + tg', f + tg, x))_x \; dx\\
    &\qquad \qquad + \int_a^b gL_z(f' + tg', f + tg, x)\; dx\\
    &= \int_a^b g [-(L_p(f' + tg', f+ tg, x))_x + L_z(f' + tg', f + tg, x)]\; dx\\
\end{align*}
Setting $t = 0$ we get 
\[h'(0) = \int_a^b g [-(L_p(f, f, x))_x + L_z(f', f, x)]\; dx = 0\]
Because $g(a) = g(b) = 0$, by a Useful Fact, we know that 
\[-(L_p(f, f, x))_x + L_z(f', f, x) = 0\]
Which is the Euler-Lagrange Equation we sought. $\qed$
\color{black}
\pagebreak

\section*{Problem 2:} Same setting as in above, and moreover assume
that for all real numbers $\xi$ and $\eta$ we have
\[L_{pp}(f', f, x)\xi^2 + 2L_{pz}(f', f, x)\xi \eta + L_{zz}(f' ,f, x)\eta^2 \geq 0\] 

This is sometimes called L is convex

Calculate $h''(t)$ where $h$ is as in your proof of the E-L equation and set $t = 0$ to show that if $L$ is convex, then \[h''(0) \geq 0\]
This means that if L is convex and f solves the Euler-Lagrange equation (= critical point of h) then f has to be a minimizer.

\color{blue}
From above, 
\[h'(t) = \int_a^b g [-(L_p(f' + tg', f+ tg, x))_x + L_z(f' + tg', f + tg, x)]\; dx\]
So taking the t-derivative:
\begin{align*}
    h''(t) &= \frac{d}{dt}\int_a^b g'L_p(f' + tg', f + tg, x) + gL_z(f' + tg', f + tg, x)\; dx\\
    &= \int_a^b \frac{d}{dt}g'L_p(f' + tg', f + tg, x)\; dx\\
    &\qquad \qquad + \int_a^b \frac{d}{dt}gL_z(f' + tg', f + tg, x)\; dx\\
    &= \int_a^b g' \cdot \frac{d}{dt} L_p(f' + tg', f + tg, x)\; dx\\
    &\qquad \qquad + \int_a^b g\cdot \frac{d}{dt}L_z(f' + tg', f + tg, x)\; dx
\end{align*}
Looking at the first lagrangian derivative:
\begin{align*}
    \frac{d}{dt} L_p(f' + tg', f + tg, x) &= L_{pp}(f' + tg', f + tg, x)(\frac{d}{dt})(f' + tg') \\
    &\qquad \qquad + L_{pz}(f' + tg', f + tg, x)(\frac{d}{dt})(f + tg)\\
    &\qquad \qquad + L_{px}(f' + tg', f + tg, x)(\frac{d}{dt})(x)\\
    &= g'L_{pp}(f' + tg', f + tg, x) + gL_{pz}(f' + tg', f + tg, x)
\end{align*}
Now the second lagrangian:
\begin{align*}
    \frac{d}{dt}L_z(f' + tg', f + tg, x) &= L_{pz}(f' + tg', f + tg, x)(\frac{d}{dt} f'+ tg')\\
    &\qquad \qquad + L_{zz}(f' + tg', f + tg, x)(\frac{d}{dt} f+ tg)\\
    &\qquad \qquad + L_{zx}(f' + tg', f + tg, x)(\frac{d}{dt} x)\\
    &= g'L_{zp}(f' + tg', f + tg, x) + g L_{zz}(f' + tg', f + tg, x)
\end{align*}
Combining, we have 
\begin{align*}
    h''(t) &= \int_a^b g'(g'L_{pp}(f' + tg', f + tg, x) + gL_{pz}(f' + tg', f + tg, x))\\
    &\qquad \qquad + g(g'L_{zp}(f' + tg', f + tg, x) + g L_{zz}(f' + tg', f + tg, x))\; dx
\end{align*}
Setting $t=0$:
\[h''(0) = \int_a^b (g')^2 L_{pp}(f', f, x) + 2g'gL_{pz}(f', f, x) + (g)^2 L_{zz}(f', f, x)\; dx\]
Then denote $\xi = g'(x)$ and $\eta = g(x)$ for all x. Thus, 
\[h''(0) = \int_a^b L_{pp}(f', f, x)\xi^2 + 2L_{pz}(f', f, x)\xi \eta + L_{zz}(f' ,f, x)\eta^2\; dx\]
But if $L$ is convex, then the integrand is non-negative by definition so 
\[h''(0) \geq 0 \qed\]
\color{black}
\pagebreak

\section*{Problem 3:}
This time in 2 dimensions, suppose the Lagrangian $L = L(p, q, z, x, y)$ is given and consider the min problem
\[\min I[u] = \int_{\Omega} L(u_x, u_y, u, x, y)\; dx\, dy\]
Show that the Euler-Lagrange equation in that case is
\[-(L_p)_x - (L_q)_y + L_z = 0\]
Where the expression is evaluated at $(u_x, u_y, u, x, y)$

Hint: Here instead of $g$ you take an arbitrary function $v$ such that $v = 0$ in $\partial \Omega$. You will also need the following integration by parts formula, valid for all $v$ with $v = 0$ on $\partial \Omega$:
\[\int_{\Omega} u_x v \; dx\, dy = -\int_{\Omega}uv_x\; dx\, dy\]
And similar for $u_y$. Assume the Useful Fact from lecture is still true in higher dimensions.

\color{blue}
Let $v$ be a function where $v = 0$ for all x in $\partial \Omega$. Then observe 
\[h(t) = I[u + tv] = \int_\Omega L(u_x + tv_x, u_y + tv_y, u + tv, x, y)\; dx\, dy\]
Then as above, $h(0) = I[u]$ and $h'(0) = 0$. Taking the derivative:
\begin{align*}
    h'(t) &= \frac{d}{dt}\int_\Omega L(u_x + tv_x, u_y + tv_y, u + tv, x, y)\; dx\, dy\\
    &= \int_\Omega \frac{d}{dt} L(u_x + tv_x, u_y + tv_y, u + tv, x, y)\; dx\, dy\\
    &= \int_{\Omega} v_xL_{p}(u_x + tv_x, u_y + tv_y, u + tv, x, y)\\
    &\qquad \qquad + v_y L_{q}(u_x + tv_x, u_y + tv_y, u + tv, x, y)\\
    &\qquad \qquad + vL_{z}(u_x + tv_x, u_y + tv_y, u + tv, x, y)\; dx\, dy
\end{align*}
Then as $v = 0$ on $\partial \Omega$,
\[\int_{\Omega} u_x v \; dx\, dy = -\int_{\Omega}uv_x\; dx\, dy\]
we have 
\begin{align*}
    h'(t) &= \int_{\Omega} -v(L_{p}(u_x + tv_x, u_y + tv_y, u + tv, x, y))_x\\
    &\qquad \qquad -v(L_{q}(u_x + tv_x, u_y + tv_y, u + tv, x, y))_y\\
    &\qquad \qquad + vL_{z}(u_x + tv_x, u_y + tv_y, u + tv, x, y)\; dx\, dy 
\end{align*}

With $t = 0$:
\[\!\!\!\!\!\!\!\!\!\!\!\!h'(0) = \int_{\Omega} v[(-L_p(u_x, u_y, u, x, y))_x + (-L_q(u_x, u_y, u, x, y))_y + (L_z(u_x, u_y, u, x, y))]\; dx\,dy = 0\]
As $v=0$ everywhere on the boundary, the Useful Fact holds and we know that 
\[(-L_p(u_x, u_y, u, x, y))_x + (-L_q(u_x, u_y, u, x, y))_y + (L_z(u_x, u_y, u, x, y)) = 0\]
which is simply the Euler-Lagrange equation 
\[-(L_p)_x - (L_q)_y + L_z = 0\]
evaluated at $(u_x, u_y, u, x, y) \qed$
\color{black}


\pagebreak
\section*{Problem 4:} 
Find the Euler-Lagrange equations of the following minimization problems. You're allowed to directly use the Euler-Lagrange equations here, no need to reprove it!

\begin{enumerate}
    \item \[\min I[u] = \int_{\Omega} \frac{1}{2}||\nabla u||^2 - F(u)\; dx\, dy\]
    Where F is an antiderivative of a given function f. Here $F(u)$ means “F of u” not “F times u”

    \color{blue}
    \[\int_{\Omega} \frac{1}{2}||\nabla u||^2 - F(u)\; dx\, dy = \int_{\Omega} \frac{1}{2}(u_x)^2 + \frac{1}{2}(u_y)^2 - F(u)\; dx\, dy\]
    So 
    \[L(p, q, z, x, y) = \frac{1}{2}p^2 + \frac{1}{2}q^2 - F(z)\]
    Which means that 
    \begin{gather*}
        L_p = p\\
        L_q = q\\
        L_z = -f(z)
    \end{gather*}
    So by the Euler-Lagrange equation $-(L_p)_x - (L_q)_y + L_z = 0$
    \[-(u_x)_x - (u_y)_y -f(u) = 0\]
    or 
    \[\boxed{-(u_{xx} + u_{yy}) = f(u)}\]
    \color{black}
    \pagebreak

    \item Here $w$ and $f$ are given 
    \[\min I[u] = \int_{\Omega}e^{-w(x, y)} \left(\frac{1}{2}||\nabla u||^2 - uf(x, y)\right)\; dx\, dy\]

    \color{blue}
    \[\min I[u] = \int_{\Omega}e^{-w(x, y)} \left(\frac{1}{2}(u_x)^2 + \frac{1}{2}(u_y)^2 - uf(x, y)\right)\; dx\, dy\]
    So 
    \[L(p, q, z, x, y) = L(u_x, u_y, u, x, y) = \frac{1}{2}u_x^2 e^{-w(x, y)} + \frac{1}{2}u_y^2e^{-w(x, y)} - uf(x, y)\]
    and as the Euler-Lagrange equation is $-(L_p)_x -(L_q)_y + L_z = 0$, we have 
    \begin{align*}
        L_p &= u_xe^{-w(x, y)}\\
        L_q &= u_ye^{-w(x, y)}\\
        L_z &= -f(x, y)e^{-w(x, y)}
    \end{align*} 
    so 
    \[-(u_xe^{-w(x, y)})_x - (u_ye^{-w(x, y)})_y- f(x, y)e^{-w(x, y)} = 0\]  
    \[-(u_{xx}e^{-w} - u_xw_xe^{-w}) - (u_{yy}e^{-w} - u_yw_y e^{-w}) = f(x, y)e^{-w(x, y)}\]
    \[\boxed{-u_{xx} - u_{yy} + u_xw_x + u_y w_y = f(x, y)}\]
    \color{black}
\end{enumerate}

\pagebreak 
\section*{Problem 5:} Consider the following system
called the Brusselator model
\[\begin{cases}
    u_t = D_1u_{xx} + a - (b + 1)u + u^2v\\
    v_t = D_2v_{xx} + bu - u^2v
\end{cases}\]
Where $a, b, D_1, D_2$ are positive constants 

\begin{enumerate}
    \item Find all constants $u_*$ and $v_*$ for which 
    \[\begin{cases}
        u(x, t) = u_*\\
        v(x, t) = v_*
    \end{cases}\]
    solves the PDE above 

    \color{blue}
    For $u$ and $v$ having constant solutions, the PDE above becomes 
    \[\begin{cases}
        0 = a - (b+1)u_* + u_*^2 v_*\\
        0 = bu_* - u_*^2 v_*
    \end{cases}\]
    Taking the second equation:
    \[(b - u_* v_*)u_* = 0 \implies u_* = \{0,\; \frac{b}{v_*}\}\]

    \emph{Case 1: $u_* = 0$}

    The first equation becomes $a = 0$. But $a$ is a strictly positive constant so this is a contradiction.\\
    
    \emph{Case 2: $u_* = \frac{b}{v_*}$}

    Here, the first equation becomes 
    \[0 = a - \frac{b(b + 1)}{v_*} + \frac{b^2}{v_*} = a + \frac{-b^2 - b + b^2}{v_*} = a - \frac{b}{v_*} \implies b = av_* \implies v_* = \frac{b}{a}\]
    Substituting this back into the form of $u_*$ then implies that $u_* = a$
    Thus for all values of $a, b, D_1, D_2$:
    \[\boxed{\begin{cases}
        u(x, t) = u_*  = a\\
        v(x, t) = v_* = \frac{b}{a}
    \end{cases}}\]
    \color{black}
    \pagebreak

    \item  The linearization of the PDE about $(u_*, v_*)$ is
    \[\begin{bmatrix}
        u_t\\v_t
    \end{bmatrix} = \begin{bmatrix}
        D_1u_{xx}\\D_2v_{xx}
    \end{bmatrix} + \begin{bmatrix}
        -(b + 1)+2u_* v_* & (u_*)^2\\
        b - 2u_* v_* & -(u_*)^2
    \end{bmatrix} \begin{bmatrix}
        u\\v
    \end{bmatrix}\]
    Where $u_*$ and $v_*$ are the values you found in (a)
    (don't prove this, but it's basically the Jacobian of the above PDE evaluated at $u_*$ and $v_*$ )
    
    Suppose the solutions $u(x, t)$ and $v(x, t)$ are of the form
    \[\begin{cases}
        u(x, t) = e^{\lambda t}\cos(\kappa x)u_0\\
        v(x, t) = e^{\lambda t}\cos(\kappa x)v_0
    \end{cases}\]
    Plug $u$ and $v$ into the linearized PDE and find a matrix $A$ depending on $\kappa$ such that
    \[A \begin{bmatrix}
        u_0\\v_0
    \end{bmatrix} = \lambda \begin{bmatrix}
        u_0\\v_0
    \end{bmatrix}\]

    \color{blue}
    \begin{align*}
        \begin{bmatrix}
            u_t\\v_t
        \end{bmatrix} &= \begin{bmatrix}
            D_1u_{xx}\\D_2v_{xx}
        \end{bmatrix} + \begin{bmatrix}
            -(b + 1)+2(a)(\frac{b}{a}) & (a)^2\\
            b - 2(a)(\frac{b}{a}) & -(a)^2
        \end{bmatrix} \begin{bmatrix}
            u\\v
        \end{bmatrix}\\
        \begin{bmatrix}
            u_t\\v_t
        \end{bmatrix} &= \begin{bmatrix}
            D_1u_{xx}\\D_2v_{xx}
        \end{bmatrix} + \begin{bmatrix}
            b - 1 & a^2\\
            -b & -a^2
        \end{bmatrix} \begin{bmatrix}
            u\\v
        \end{bmatrix}
    \end{align*}
    \[\begin{cases}
            u(x, t) = e^{\lambda t}\cos(\kappa x)u_0\\
            v(x, t) = e^{\lambda t}\cos(\kappa x)v_0
        \end{cases} \implies 
    \parbox{2in}{
        \begin{align*}
            &\begin{cases}
                u_t = \lambda e^{\lambda t}\cos(\kappa x)u_0\\
                v_t = \lambda e^{\lambda t}\cos(\kappa x)v_0
            \end{cases}\\
            &\begin{cases}
                u_{xx} = -\kappa^2 e^{\lambda t}\cos(\kappa x)u_0\\
                v_{xx} = -\kappa^2 e^{\lambda t}\cos(\kappa x)v_0
            \end{cases}
        \end{align*}
    }\]

    \begin{align*}
        \begin{bmatrix}
            \lambda e^{\lambda t}\cos(\kappa x)u_0\\
            \lambda e^{\lambda t}\cos(\kappa x)v_0
        \end{bmatrix} &= \begin{bmatrix}
            -D_1\kappa^2 e^{\lambda t}\cos(\kappa x)u_0\\
            -D_2\kappa^2 e^{\lambda t}\cos(\kappa x)v_0
        \end{bmatrix} + \begin{bmatrix}
            b - 1 & a^2\\
            -b & -a^2
        \end{bmatrix} \begin{bmatrix}
            e^{\lambda t}\cos(\kappa x)u_0\\
            e^{\lambda t}\cos(\kappa x)v_0
        \end{bmatrix}\\
        \lambda e^{\lambda t}\cos(\kappa x) \begin{bmatrix}
            u_0\\v_0
        \end{bmatrix} &= e^{\lambda t}\cos(\kappa x) \left(\begin{bmatrix}
            -D_1 \kappa^2 u_0\\
            -D_2 \kappa^2 v_0
        \end{bmatrix} + \begin{bmatrix}
            b-1 & a^2\\
            -b & -a^2
        \end{bmatrix} \begin{bmatrix}
            u_0\\
            v_0
        \end{bmatrix}\right)
    \end{align*}
    For $u, v \neq 0$:
    \begin{align*}
        \lambda \begin{bmatrix}
            u_0\\v_0
        \end{bmatrix} &= \begin{bmatrix}
            -D_1 \kappa^2 u_0\\
            -D_2 \kappa^2 v_0
        \end{bmatrix} + \begin{bmatrix}
            (b-1)u_0 + a^2v_0\\
            -bu_0 -a^2v_0
        \end{bmatrix}\\
        &= \begin{bmatrix}
            -D_1 \kappa^2 u_0 + (b-1)u_0 + a^2v_0\\
            -D_2 \kappa^2 v_0 -bu_0 -a^2v_0
        \end{bmatrix}\\
        &= \begin{bmatrix}
            (-D_1\kappa^2 + b - 1)u_0 + a^2 v_0\\
            -bu_0 + (-D_2\kappa^2 - a^2)v_0
        \end{bmatrix}\\
        &= \begin{bmatrix}
            -D_1\kappa^2 + b - 1 & a^2\\
            -b & -D_2\kappa^2 - a^2
        \end{bmatrix}\begin{bmatrix}
            u_0\\
            v_0
        \end{bmatrix}
    \end{align*}
    Thus 
    \[A \begin{bmatrix}
        u_0\\v_0
    \end{bmatrix} = \lambda \begin{bmatrix}
        u_0\\v_0
    \end{bmatrix}\]
    for 
    \[\boxed{A = \begin{bmatrix}
        -D_1\kappa^2 + b - 1 & a^2\\
        -b & -D_2\kappa^2 - a^2
    \end{bmatrix}}\]
    \color{black}
\end{enumerate}


\end{document}