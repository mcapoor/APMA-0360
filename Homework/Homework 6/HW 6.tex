\documentclass[12pt]{article} 
\usepackage[utf8]{inputenc}
\usepackage{geometry}
\geometry{letterpaper}
\usepackage{graphicx} 
\usepackage{parskip}
\usepackage{booktabs}
\usepackage{array} 
\usepackage{paralist} 
\usepackage{verbatim}
\usepackage{subfig}
\usepackage{fancyhdr}
\usepackage{sectsty}

\pagestyle{fancy}
\renewcommand{\headrulewidth}{0pt} 
\lhead{}\chead{}\rhead{}
\lfoot{}\cfoot{\thepage}\rfoot{}

%%% SECTION TITLE APPEARANCE
\allsectionsfont{\sffamily\mdseries\upshape} 

%%% ToC (table of contents) APPEARANCE
\usepackage[nottoc,notlof,notlot]{tocbibind} 
\usepackage[titles,subfigure]{tocloft}
\renewcommand{\cftsecfont}{\rmfamily\mdseries\upshape}
\renewcommand{\cftsecpagefont}{\rmfamily\mdseries\upshape} %

\usepackage{amsmath}
\usepackage{amssymb}
\usepackage{empheq}
\usepackage{xcolor}
\renewcommand{\L}[1]{\mathcal{L}\{#1\}}
\newcommand{\ans}[1]{\boxed{\text{#1}}}
\newcommand{\vecs}[1]{\langle #1\rangle}
\renewcommand{\hat}[1]{\widehat{#1}}
\newcommand{\F}[1]{\mathcal{F}(#1)}
\renewcommand{\P}{\mathbb{P}}
\newcommand{\R}{\mathbb{R}}
\newcommand{\qed}{\quad \blacksquare}
\newcommand{\brak}[1]{\langle #1 \rangle}

\title{APMA 0360: Homework 6}
\author{Milan Capoor}
\date{17 March 2023}

\begin{document}
\maketitle
Note: you do not need to show work for the three cases for boundary value problems except in problem 1.

\section*{Problem 1: Wave equation with mixed conditions} 
\[\begin{cases}
    u_{tt}= c^2 u_{xx}\\
    u_x(0, t) = 0\\
    u(2, t) = 0
\end{cases}\]
\color{blue}
Assume that $u(x, t) = X(x)T(t)$. Then 
\[( X(x)T(t))_{tt} = c^2( X(x)T(t))_{xx}\]
\[XT'' = c^2X''T\]
\[\frac{T''}{c^2T} = \frac{X''}{X}\]
Then note that 
\[\left(\frac{X''}{X}\right)_t = 0\]
and 
\[\left(\frac{X''}{X}\right)_x = \left(\frac{T''}{c^2T}\right)_x = 0\]
so 
\[\frac{T''}{c^2T} = \frac{X''}{X} = \lambda\]
for some constant $\lambda$ giving a system of ODE:
\[\begin{cases}
    X'' = \lambda X\\
    T'' = c^2T\lambda
\end{cases}\]
Using the boundary conditions:
\begin{align*}
    u(2, t) = 0 \implies X(2)T(t) = 0 \implies X(2) = 0\\
    u_x(0, t) = 0 \implies X'(0)T(t) = 0 \implies X'(0) = 0
\end{align*}
Which gives an ODE for X:
\[\begin{cases}
    X'' = \lambda X\\
    X(2) = 0\\
    X'(0) = 0
\end{cases}\]
Auxiliary equation:
\[r^2 = \lambda \implies r = \pm \omega\]
Case 1: $\lambda > 0$
\[X(x) = Ae^{\omega x} + Be^{-\omega x}\]
\[X'(0) = A\omega e^{\omega (0)} + B\omega e^{\omega (0)} = A\omega + B\omega = 0 \implies B = -A\]
\[X(x) = Ae^{\omega x} - Ae^{-\omega x}\]
\[X(2) = Ae^{2\omega} - Ae^{-2\omega} = 0 \implies e^{2\omega} = e^{2\omega} \implies -2\omega = 2\omega \implies \omega = 0\]
So there are no nonzero solutions

Case 2: $\lambda = 0$
\[X(x) = A + Bx\]
\[X(2) = A + 2B = 0 \implies A = -2B\]
\[X(x) = -2B + Bx\]
\[X'(x) = B \implies X'(0) = B = 0 \implies X(x) = 0\]
So there are no nonzero solutions here either 

Case 3: $\lambda < 0$
\[X(x) = A\cos(\omega x) + B\sin(\omega x)\]
\[X'(x) = -A\omega\sin(\omega x) + B\omega \cos(\omega x)\]
\[X'(0) = B\omega = 0\implies B = 0\]
\[X(x) = A\cos(\omega x)\]
\[X(2)= A\cos(2\omega) = 0 \implies \cos(2\omega) = 0\]
So $\omega = \frac{\pi}{4} + \frac{\pi}{2}m \quad (m = 0, 1, 2,...)$ and the eigenvalues are 
\[\lambda = -(\frac{\pi}{4} + \frac{\pi}{2}m)^2\]
corresponding to the eigenfunctions
\[X(x) = \cos((\frac{\pi}{4} +\frac{\pi}{2}m)x)\]
So back to the T equation:
\[ T'' = -c^2(\frac{\pi}{4} + \frac{\pi}{2}m)^2T\] 
\[z^2 + c^2(\frac{\pi}{4} + \frac{\pi}{2}m)^2 = 0\]
\[z = \pm i(c(\frac{\pi}{4} + \frac{\pi}{2}m))\]
\[T(t) = A\cos(c(\frac{\pi}{4} + \frac{\pi}{2}m)) + B\sin(c(\frac{\pi}{4} + \frac{\pi}{2}m))\]
so 
\[\boxed{u(x, t) =\sum_{m=0}^\infty A_m\cos(c(\frac{\pi}{4} + \frac{\pi}{2}m)) + B_m\sin(c(\frac{\pi}{4} + \frac{\pi}{2}m))\cos((\frac{\pi}{4} +\frac{\pi}{2}m)x) }\]
\color{black}
\pagebreak
\section*{Problem 2: Wave equation with friction} 
\[\begin{cases}
    u_{tt} + ru_t = c^2 u_{xx}\\
    u(0, t) = 0\\
    u(\pi, t) = 0
\end{cases}\]
Note: Assume $0 < r < 2c$

\color{blue}
Assume $u(x, t) = X(x)T(t)$ so 
\[(X(x)T(t))_{tt} + r(X(x)T(t))_t = c^2(X(x)T(t))_{xx}\]
\[XT'' + rXT' = c^2X''T\]
\[\frac{T'' + rT'}{c^2T} = \frac{X''}{X}\]
\[\frac{T''}{c^2T} + \frac{rT'}{c^2T} = \frac{X''}{X}\]
\[\left(\frac{T''}{c^2T} + \frac{rT'}{c^2T}\right)_x + \left(\frac{X''}{X}\right)_t = 0\]
\[\frac{T''}{c^2T} + \frac{rT'}{c^2T} = \frac{X''}{X} = \lambda\]
Looking at the X equation:
\[X'' = \lambda X\]
\[u(0, t) = X(0)T(t) = 0 \implies X(0) = 0\]
\[u(\pi, t) = 0 \implies X(\pi) = 0\]
\[r^2 = \lambda \implies r = \pm \omega\]
Looking at $\lambda < 0$:
\[X(x) = A\cos(\omega x) + B\sin(\omega x)\]
\[X(0) = A = 0 \implies X(x) = B\sin(\omega x)\]
\[X(\pi) = B \sin(\pi \omega) = 0 \implies \sin(\pi \omega) = 0\]
so $\omega = m \quad (m = 0, 1, 2, ...)$
and the eigenvalues are 
\[\lambda = -m^2 \quad (m = 0, 1, 2, ...)\]
corresponding to eigenfunction 
\[X(x) = \sin(mx)\]

Going back to the T equation:
\[T'' + rT' = c^2T\lambda\]
\[T'' + rT' = -c^2m^2T\]
So we have auxiliary equation 
\[s^2 + rs + c^2m^2 = 0 \implies s = \frac{-r \pm \sqrt{r^2 - 4c^2m^2}}{2}\]
However, as $r < 2c$ we know that $r^2 < 4c^2 \implies r^2 - 4c^2m^2 < 0$ so the solutions are complex. Then,
\[ s = \frac{-r \pm i\sqrt{4c^2m^2 - r^2}}{2}\]
and 
\[T(t) = e^{-rt}\left(A\cos\left(\frac{\sqrt{4c^2m^2 - r^2}}{2}x\right) + B\sin\left(\frac{\sqrt{4c^2m^2 - r^2}}{2}x\right)\right)\]
so 
\[u(x, t) = e^{-rt}\left(A\cos\left(\frac{\sqrt{4c^2m^2 - r^2}}{2}x\right) + B\sin\left(\frac{\sqrt{4c^2m^2 - r^2}}{2}x\right)\right)\sin(mx)\]
which by linearity gives us 
\[\boxed{u(x, t) = \sum_{m = 0}^\infty e^{-rt}\left(A_m\cos\left(\frac{\sqrt{4c^2m^2 - r^2}}{2}x\right) + B_m\sin\left(\frac{\sqrt{4c^2m^2 - r^2}}{2}x\right)\right)\sin(mx)}\]
\color{black}
\pagebreak
\section*{Problem 3:}
\[\begin{cases}
    tu_{t}= u_{xx} + 2u\\
    u(0, t) = 0\\
    u(\pi, t) = 0
\end{cases}\]
Note: Define $\lambda' = \lambda - 2$

\color{blue}
Let $u(x, t) = X(x)T(t)$. Then 
\[t(X(x)T(t))_t = (X(x)T(t))_{xx} + 2(X(x)T(t))\]
\[tXT' = X''T + 2XT\]
\[\frac{tT'}{T} = \frac{X''}{X} + 2\]
\[\frac{X''}{X} = \frac{tT'}{T} + 2 = \lambda\]
for some constant $\lambda$ or with $\lambda' = \lambda - 2$,
\[\frac{X''}{X} = \frac{tT'}{T} = \lambda'\] 
Then 
\[\begin{cases}
    X'' = \lambda' X\\
    tT' = \lambda' T
\end{cases}\]
Then,
\[u(\pi, t) = 0 \implies X(\pi)T(t) = 0 \implies X(\pi) = 0\]
\[u(0, t) = 0\implies X(0)T(t) = 0 \implies X(0) = 0\]
Giving us an ODE for X:
\[\begin{cases}
    X'' = \lambda' X\\
    X(\pi) = 0\\
    X(0) = 0
\end{cases}\]
From the auxiliary equation, with $\lambda' < 0$
\[r^2 = \lambda \implies r = \pm \omega i \]
\[X(x) = A\cos(\omega x)+ B\sin(\omega x)\]
\[X(0) = A = 0 \implies X(x) = B\sin(\omega x)\]
\[X(\pi) = B\sin(\pi \omega) = 0 \implies \sin(\pi \omega) = 0 \implies \omega = m \quad (m = 0, 1, 2, ...)\]
So 
\[\lambda' = -m^2 \quad (m = 0, 1, ...)\]
\[X(x) = \sin(mx)\]
Then 
\[tT' = -m^2T\]
\[T = e^{\int \frac{m^2}{t}\; dt} = t^{-m^2}\]
and 
\[u(x, t) = \sum_{m=0}^\infty A_m\sin(mx) t^{- m^2}\]
Then making the going back to $\lambda$ from $\lambda'$:
\[\boxed{u(x, t) = \sum_{m=1}^\infty A_m\sin((m - \sqrt{2})x) t^{2-m^2}}\]
\color{black}
\pagebreak
\section*{Problem 4:}
\begin{enumerate}
    \item Solve 
    \[\begin{cases}
        u_{tt} = c^2u_{xx}\\
        u(0, t) = 0\\
        u(1, t) = 0\\
        u(x, 0) = 3\sin(2\pi x)\\
        u_t(x, 0) = 0
    \end{cases}\]
    Note: you are allowed to directly use the formula from lecture

    \color{blue} 
    From lecture,
    \[u(x, t) = \sum_{m=1}^\infty (A_m\cos(\pi m ct) + B_m\sin(\pi mct)) \sin(\pi m x)\]
    Then using the new initial conditions,
    \[u(x, 0) = \sum_{m=1}^\infty A_m\sin(\pi m x) = 3\sin(2\pi x)\]
    Which suggests that $A_2 = 3$ and $A_m = 0$ for all other values of m. And from the other initial condition, 
    \[u_t(x, t) = \sum_{m=1}^\infty (-A_m(\pi mc)\sin(\pi m ct) + B_m(\pi mc)\cos(\pi mct))\sin(\pi m x)\]
    \[u_t(x, 0) = \sum_{m=1}^\infty B_m(\pi mc)\sin(\pi m x) = 0\]
    which implies that $B_m$ equals 0 for all m. So at last, we have 
    \[\boxed{u(x, t) = 3\cos(2\pi x)\sin(2\pi x)}\]
    \color{black}
    \item Use (1) to solve 
    \[\begin{cases}
        u_{tt} = c^2u_{xx}\\
        u(0, t) = 5\\
        u(1, t) = 5\\
        u(x, 0) = 3\sin(2\pi x) + 5\\
        u_t(x, 0) = 0
    \end{cases}\]

    \color{blue}
    Let $v(x, t) = u(x, t) - 5$. Then 
    \[v_{tt} = (u - 5)_{tt} = u_{tt} = c^2u_{xx} = c^2(v + 5)_{xx} = c^2v_{xx}\]
    So 
    \[\begin{cases}
        v_{tt} = c^2v_{xx}
        v(0, t) = u(0, t) - 5 = 0\\
        v(1, t) = u(1, t) - 5 = 0\\
        v(x, 0) = u(x, 0) - 5 = 3\sin(2\pi x)\\
        v_t(x, 0) = u_t(x, 0) = 0
   \end{cases}\]
    But this system is exactly the same as from part 1! So 
    \[v(x, t) = 3\cos(2\pi x)\sin(2\pi x)\]
    And as, $u = v + 5$, 
    \[\boxed{u(x, t) = 3\cos(2\pi x)\sin(2\pi x) + 5}\]
   \color{black}

    \item Use (1) to solve 
    \[\begin{cases}
        u_{tt} = c^2u_{xx}\\
        u(0, t) = 3\\
        u(1, t) = 5\\
        u(x, 0) = 3\sin(2\pi x) + (2x + 3)\\
        u_t(x, 0) = 0
    \end{cases}\]
    \color{blue}
    Let $v(x, t) = u(x, t) - f(x)$ where f is a linear function such that $f(0) = 3$ and $f(1) = 5$. Then 
    \[f(x) = \frac{5 -3}{1 - 0}(x - 0) + 3 = 2x + 3\]
    and 
    \[v(x, t) = u(x, t) - 2x - 3\]
    Which gives us a new PDE
    \[\begin{cases}
        v_{tt} = c^2v_{tt}\\
        v(0, t) = 0\\
        v(l, t) = 0\\
        v(x, 0) = 3\sin(2\pi x) + (2x + 3) - (2x+3) = 3\sin(2\pi x)\\
        v_t(x, 0) = u_t(x, 0) - 2(0) = 0
    \end{cases}\]
    Once again, this corresponds to the solution from part 1 so 
    \[v(x, t) = 3\cos(2\pi x)\sin(2\pi x)\]
    and 
    \[\boxed{u(x, t) = 3\cos(2\pi x)\sin(2\pi x) + 2x + 3}\]
\end{enumerate}
\pagebreak
\section*{Problem 5: The Infinity Laplacian}
\[(u_t)^2 (u_{tt}) + 2(u_x)(u_{t})(u_{xt}) + (u_x)^2(u_{xx}) = 0\]
\begin{enumerate}
    \item For separation of variables use $u(x, t) = X(x) + T(t)$ to show that 
    \[(T')^2T'' = -(X')^2 X''\]

    \color{blue}
    Let $u(x, t) = X(x) + T(t)$ so 
    \begin{align*}
        ((X(x) &+ T(t))_t)^2 ((X(x) + T(t))_{tt})\\ 
        &+ 2((X(x) + T(t))_x)((X(x) + T(t))_{t})((X(x) + T(t))_{xt}) \\
        &+ ((X(x) + T(t))_x)^2((X(x) + T(t))_{xx}) = 0
    \end{align*}
    \[(T')^2(T'') + 2(X')(T')(0)+ (X')^2(X'') = 0\]
    \[(T')^2T'' + (X')^2(X'') = 0\]
    \[(T')^2T'' = -(X')^2(X'') \quad \blacksquare\]
    \color{black}
    \item Show both sides are constant, equal to $\lambda$. For simplicity assume $\lambda = 1/3$. Use this to solve for X and T and then for u(x, t) Assume that any constants of integration are 0.
    
    \color{blue}
    \[((T')^2T'')_x = 0\]
    \[((T')^2T'')_t = (-(X')^2(X''))_t = 0\] 
    So both sides are constant and 
    \[(T')^2T'' = -(X')^2(X'') = \lambda = \frac{1}{3}\]
    This gives us a system of ODEs
    \[\begin{cases}
        (T')^2T'' = \frac{1}{3}\\
        -(X')^2(X'') = \frac{1}{3}
    \end{cases}\]

    Starting with T, let $v(t) = T'(t)$ so 
    \begin{align*}
        v^2v' &= \frac{1}{3}\\
        v^2 \frac{dv}{dt} &= \frac{1}{3}\\
        v^2 \; dv &= \frac{1}{3} \; dt\\
        \frac{1}{3}v^3&= \frac{1}{3}t + C_1\\
        v &= \sqrt[3]{t + C_1}
    \end{align*}
    So 
    \[T' = (t + C_2)^\frac{1}{3}\]
    \[T = \frac{3}{4}(t + C_1)^{\frac{4}{3}} + C_2\]
    but assuming that all the constants of integration are 0,
    \[T = \frac{3}{4}t^{\frac{4}{3}}\]
    Similarly, for x let $w(x) = X'(x)$ so 
    \begin{align*}
        w^2 w' &= -\frac{1}{3}\\
        \frac{w^3}{3} &= -\frac{1}{3}x + C_1\\
        w &= \sqrt[3]{-x + C_1}
    \end{align*}
    \[X' = (-x + C_1)^{\frac{1}{3}} \implies X = \frac{3}{4}(-x + C_1)^{\frac{4}{3}}+ C_2 \implies X = \frac{3}{4}x^{\frac{4}{3}}\]
    so 
    \[\boxed{u(x, t) = \frac{3}{4}(t^\frac{4}{3} + x^\frac{4}{3})}\]
\end{enumerate}

\end{document}