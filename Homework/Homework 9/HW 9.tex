\documentclass[12pt]{article} 
\usepackage[utf8]{inputenc}
\usepackage{geometry}
\geometry{letterpaper}
\usepackage{graphicx} 
\usepackage{parskip}
\usepackage{booktabs}
\usepackage{array} 
\usepackage{paralist} 
\usepackage{verbatim}
\usepackage{subfig}
\usepackage{fancyhdr}
\usepackage{sectsty}

\pagestyle{fancy}
\renewcommand{\headrulewidth}{0pt} 
\lhead{}\chead{}\rhead{}
\lfoot{}\cfoot{\thepage}\rfoot{}


%%% ToC (table of contents) APPEARANCE
\usepackage[nottoc,notlof,notlot]{tocbibind} 
\usepackage[titles,subfigure]{tocloft}
\renewcommand{\cftsecfont}{\rmfamily\mdseries\upshape}
\renewcommand{\cftsecpagefont}{\rmfamily\mdseries\upshape} %

\usepackage{amsmath}
\usepackage{amssymb}
\usepackage{empheq}
\usepackage{xcolor}
\renewcommand{\L}[1]{\mathcal{L}\{#1\}}
\newcommand{\ans}[1]{\boxed{\text{#1}}}
\newcommand{\vecs}[1]{\langle #1\rangle}
\renewcommand{\hat}[1]{\widehat{#1}}
\newcommand{\F}[1]{\mathcal{F}(#1)}
\renewcommand{\P}{\mathbb{P}}
\newcommand{\R}{\mathbb{R}}
\newcommand{\qed}{\quad \blacksquare}
\newcommand{\brak}[1]{\langle #1 \rangle}

\title{APMA 0360: Homework 9}
\author{Milan Capoor}
\date{21 April 2023 }

\begin{document}
\maketitle
\section*{Problem 1:}
Use the Laplacian in polar coordinates to find the radial solution of
\[u_{xx} + u_{yy} = 1\]
Such that $u = 0$ on the circles of radius $r = 1$ and $r = 2$

\color{blue}
In polar coordinates, the Laplacian is 
\[u_{rr} + \frac{1}{r}u_r + \frac{1}{r^2} u_{\theta \theta} = 1\] 

As the solution is radial, $u_{\theta \theta} = 0$ so 
\[u_{rr} + \frac{1}{r}u_r = 1\]
\[ru_{rr} + u_r = r\]
\[(ru_r)' = r\]
\[ru_r = \frac{1}{2}r^2 + A\]
\[u_r = \frac{1}{2}r + \frac{A}{r}\]
\[u(r) = \frac{r^2}{4} + A\ln r + B\]
Applying linearity and checking initial conditions:
\[\begin{cases}
    u(1) = \frac{1}{4} + A\ln(1) + B = 0 \implies  B = -\frac{1}{4}\\
    u(2) = 1 + A_m\ln(2) + B = 0 \implies A = -\frac{3}{4\ln 2}
\end{cases}\]
so the radial solution of $\Delta u = 1$ such that $u = 0$ on $r^2 = 1$ and $r^2 = 4$ is 
\[\boxed{u(x, y) = \frac{x^2 + y^2}{4} - \frac{3}{4\ln 2} \ln(x^2 + y^2) - \frac{1}{4}}\]

\color{black}
\pagebreak

\section*{Problem 2:} 
\textbf{Note:} In the problem below, you may use without proof the fact that the Laplacian in spherical coordinates is
\[u_{xx} + u_{yy} + u_{zz} = u_{rr} + \frac{2}{r} u_r + \; ...\]
where $r = \sqrt{x^2 + y^2 + z^2}$ and the extra terms don't depend on r 

Find all radial solutions of
\[u_{xx} + u_{yy} + u_{zz} = k^2u \quad k > 0\]
\textbf{Hint:} see above. To solve the ODE, use $v = ru(r)$

\color{blue}
Let $v = ru(r)$. Then 
\[v_r = u + ru_r\]
\[v_{rr} = 2u_r + ru_{rr}\]
So 
\[\frac{v_{rr}}{r} = u_{rr} + \frac{2}{r}u_r = u_{xx} + u_{yy} + u_{zz} = k^2 u\] 
\[\frac{v_{rr}}{r} = \frac{k^2v}{r}\]
\[v_{rr} = k^2 v\]
\[v = Ae^{kr} + Be^{-kr}\]
\[ru = Ae^{kr} + Be^{-kr}\]
\[u = \frac{Ae^{kr}}{r} + \frac{Be^{-kr}}{r}\]
So 
\[\boxed{u(x, y, z) = \frac{1}{\sqrt{x^2 + y^2 + z^2}}\left(Ae^{k\sqrt{x^2 + y^2 + z^2}} + Be^{-k\sqrt{x^2 +y^2 + z^2}}\right)}\]
\color{black}
\pagebreak

\section*{Problem 3:}
Solve using separation of Variables
\[\begin{cases}
    u_{xx} + u_{yy} = 0\\
    u_y(x, 0) = 0\\
    u_y(x, \pi) = 0\\
    u(0, y) = 0\\
    u(\pi, y) = \cos^2 y = \frac{1}{2}+\frac{1}{2}\cos(2y)
\end{cases}\]

\color{blue}
Assume $u(x, y) = X(x)Y(y)$. 
\begin{gather*}
    X''Y + XY'' = 0\\
    -\frac{X''}{X} = \frac{Y''}{Y} = \lambda
\end{gather*}

\emph{Initial conditions:}
\begin{gather*}
    u_y(x, 0) = 0 \implies Y'(0) = 0\\
    u_y(x, \pi) = 0 \implies Y'(\pi) = 0
\end{gather*}

\emph{Boundary Value:}
\[Y'' = \lambda Y\]
$\lambda > 0$: 
\[Y = Ae^{\omega x} + Be^{-\omega x}\]
\[Y' = A\omega e^{\omega y} + B\omega e^{-\omega y}\]
\[Y'(0) \implies A = - B\]
\[Y'(\pi) = A\omega e^{\pi \omega} - A\omega e^{-\pi \omega} \implies \omega = -\omega = 0 \implies Y = A - A = 0\]

$\lambda = 0$:
\[Y = A + By\]
\[Y'(0) = B = 0\]
\[Y'(\pi) = 0 = 0\]
So $\lambda = 0$ is an eigenvalue with $Y = A$. Which means that 
\[X'' = 0 \implies X = A+ Bx\]
so $u(x, y) = XY = A + Bx$ for some arbitrary constants.

$\lambda < 0$:
\[Y(y) = A\cos(\omega y) + B\sin(\omega y)\]
\[Y'(y) = -A\omega \sin(\omega y) + B\omega\cos(\omega y)\]
\[Y'(0) = B\omega = 0 \implies B = 0\]
\[Y'(\pi) = -A\omega \sin(\pi \omega) = 0 \implies \sin(\pi m) = 0 \quad (m=1, 2, ...)\]
So $Y(y) = \cos(my)$ corresponding to $\lambda = -m^2$

\emph{Back to Laplace:}
\[X''(X) = -\lambda X(x) = m^2X(x)\]
\begin{align*}
    X(x) &= Ae^{mx} + Be^{-mx}\\
    &= A(\cosh(mx) + \sinh(mx)) + B(\cosh(mx)- \sinh(mx))\\
    &= (A + B)\cosh(mx) + (A-B)\sinh(mx)\\
    &= B\cosh(mx) + C\sinh(mx)
\end{align*}

Then putting this all together
\[u(x, y) = X(x)Y(y) = A_0 + B_0x + (B\cosh(mx) + C\sinh(mx) )\cos(my)\]
\[u(x, y) =  A_0 + B_0x + \sum_{m=1}^\infty(A_m\cosh(mx) + B_m\sinh(mx) )\cos(my)\]
\[u(0, y) = A_0 + \sum_{m=1}^\infty A_m\cos(my) = 0 \implies \sum_{m=0}^\infty A_m\cos(my) = 0 \implies A_m = 0\]
\[u(\pi, y) = B_0\pi + \sum_{m=1}^\infty \underbrace{B_m\sinh(\pi m)}_{\tilde{B}_m}\cos(my) = \frac{1}{2} + \frac{1}{2}\cos(2y)\]
\[\begin{cases}
    \tilde{B}_2 = \frac{1}{2} \implies B_2 = \frac{1}{2\sinh(2\pi)}\\
    \tilde{B}_m = 0 \qquad (m \neq 2)\\
    B_0\pi = \frac{1}{2} \implies B_0 = \frac{1}{2\pi}
\end{cases}\]
\[\boxed{u(x, y) = \frac{x}{2\pi} + \frac{1}{2\sinh(2\pi)}\sinh(2x)\cos(2y)}\]
\color{black}
\pagebreak

\section*{Problem 4:}
\begin{enumerate}
    \item Use the energy method to show that if u solves the heat equation in n dimensions
    \[\begin{cases}
        u_t = D\Delta u \quad \in \Omega\\
        u(x, t) = 0 \qquad x \in \partial \Omega\\
        u(x, 0) = 0 \qquad x \in \Omega
    \end{cases}\]
    Then u = 0
    \color{blue}
    \begin{align*}
        u_tu &= D\Delta u\cdot u\\
        \int_{\Omega} u_t u \; dx &= D \int_{\Omega}\Delta u\cdot u\; dx\\
        \frac{d}{dt}\int_{\Omega} \frac{1}{2}(u)^2 \; dx &= -D\int_{\Omega} \underbrace{||\nabla u||^2}_{\geq 0}\; dx\\
        \frac{d}{dt}\int_{\Omega} \frac{1}{2}(u)^2 \; dx &\leq 0
    \end{align*}
    So with $E(t) = \int_{\Omega} \frac{1}{2}(u)^2 \; dx \geq 0$, 
    \[E'(t) \leq 0 \implies 0 \leq E(t) \leq E(0) = \int_{\Omega} \frac{1}{2}(u(x, 0))^2 \; dx = 0 \implies E(t) = 0\]
    Hence, 
    \[\int_{\Omega} \frac{1}{2}(u(x, t))^2 \; dx = 0 \implies u(x, t) = 0 \quad \in \Omega\] 
    and because $u(x, t) = 0$ for $x \in \partial \Omega$, $u = 0 \qed$.
    \color{black}

    \item Use (i) to show that there is at most one solution to 
    \[\begin{cases}
        u_t = D\Delta u + f(x, t) \quad \in \Omega\\
        u(x, t) = g(x,  t) \qquad x \in \partial \Omega\\
        u(x, 0) = h(x) \qquad x \in \Omega
    \end{cases}\]
   \textbf{Hint:} Use the following integration by parts formula, valid for all $v(x, t)$ with $v(x, t) = 0$ for $x$ on $\partial \Omega$
   \[\int_{\Omega} (\Delta u) v\; dx = -\int_{\Omega}(\nabla u) \cdot (\nabla v)\; dx\]

   \color{blue}
   Let $w = u - v$ where $u$ and $v$ are solutions. Then 
   \begin{align*}
        w_t &= (u - v)_t\\
        & = D\Delta u + f(x, t) - D\Delta v - f(x, t) = D(\Delta u - \Delta v) \\
        &= D((u_{xx} - v_{xx}) + (u_{yy} - v_{yy}))\\
        &= D(w_{xx} + w_{yy})\\
        &= D\Delta w
   \end{align*}
  
   and with the initial conditions,
   \[w(x, t) = u(x, t) - v(x, t) = g(x, t) - g(x, t)= 0\]
   \[w(x, 0) = u(x, 0) - v(x, 0) = h(x) - h(x) = 0\]
   Hence, $w$ solves the system from part (i). Thus, $w = 0$ so $u - v = 0 \implies u = v$ and there is only one solution. $\qed$
   \color{black}
\end{enumerate}
\pagebreak

\section*{Problem 5:} Suppose u solves Laplace's equation
on the disk $x^2 + y^2 \leq 4$ with $u = 3 \sin(2\theta) + 1$ on $x^2 + y^2 = 4$.

Without finding the solution:
\begin{enumerate}
    \item Find the maximum value of $u$ on $x^2 + y^2 \leq 4$
    
    \color{blue}
    By the strong maximum principle, because $\Delta u =0$ inside the boundary of $x^2 + y^2 = 4$, its maximum exists on that boundary. Thus it suffices to find the maximum of $u$ on that circle. 

    Thus with $0 \leq \theta \leq 2\pi$:
    \begin{gather*}
        \frac{d}{d\theta} u = 6\cos(2\theta) = 0 \implies \theta = \{\frac{\pi}{4}, \frac{3\pi}{4}, \frac{5\pi}{4}, \frac{7\pi}{4}\}\\
        u(\frac{\pi}{4}) = 4\\
        u(\frac{3\pi}{4}) = -2\\
        u(\frac{5\pi}{4}) = 4\\
        u(\frac{7\pi}{4}) = -2
    \end{gather*}
    So the maximum value of $u$ on the disk is 
    \[\boxed{u(\sqrt{2}, \sqrt{2}) = u(-\sqrt{2}, -\sqrt{2}) = 4}\]
    \color{black}

    \item Find $u(0,0)$
    
    \textbf{Hint:} For (b) the mean value formula also holds if you integrate u on circles/spheres, that is
    \[\frac{1}{|\partial B(x, r)|} \int_{\partial B(x, r)} u(y) \; dy = u(x)\]
    Where $\partial B(x, r)$ is the circle/sphere centered at x and radius r. Integrating over a circle means integrating with respect to $\theta$

    \color{blue}
    By the Mean-Value Formula, as $\Delta u=0$ then forall $x$ and $r > 0$,
    \[\frac{1}{|\partial B(x, r)|} \int_{\partial B(x, r)} u(y) \; dy = u(x)\]
    Hence 
    \[\frac{1}{|x^2 + y^2 = 4|} \int_0^{2\pi} 2u(y)\; d\theta = u(x)\]
    \[\frac{1}{4\pi}\int_0^{2\pi} 6\sin(2\theta) + 2\; d\theta = \frac{1}{4\pi}[-3\cos(2\theta) + 2\theta]_0^{2\pi} = -\frac{3}{4\pi} + \frac{2\pi}{2\pi} + \frac{3}{4\pi} - 0 = 1 = u(x)\]
    Which means that the value at the center of the circle is $1$. Or:
    \[\boxed{u(0,0) = 1}\]
    \color{black}
\end{enumerate}



\end{document}