\documentclass[12pt]{article} 
\usepackage[utf8]{inputenc}
\usepackage{geometry}
\geometry{letterpaper}
\usepackage{graphicx} 
\usepackage{parskip}
\usepackage{booktabs}
\usepackage{array} 
\usepackage{paralist} 
\usepackage{verbatim}
\usepackage{subfig}
\usepackage{fancyhdr}
\usepackage{sectsty}

\pagestyle{fancy}
\renewcommand{\headrulewidth}{0pt} 
\lhead{}\chead{}\rhead{}
\lfoot{}\cfoot{\thepage}\rfoot{}

%%% ToC (table of contents) APPEARANCE
\usepackage[nottoc,notlof,notlot]{tocbibind} 
\usepackage[titles,subfigure]{tocloft}
\renewcommand{\cftsecfont}{\rmfamily\mdseries\upshape}
\renewcommand{\cftsecpagefont}{\rmfamily\mdseries\upshape} %

\usepackage{amsmath}
\usepackage{amssymb}
\usepackage{empheq}
\usepackage{xcolor}
\renewcommand{\L}[1]{\mathcal{L}\{#1\}}
\newcommand{\ans}[1]{\boxed{\text{#1}}}
\newcommand{\vecs}[1]{\langle #1\rangle}
\renewcommand{\hat}[1]{\widehat{#1}}
\newcommand{\F}[1]{\mathcal{F}(#1)}
\renewcommand{\P}{\mathbb{P}}
\newcommand{\R}{\mathbb{R}}
\newcommand{\qed}{\quad \blacksquare}
\newcommand{\brak}[1]{\langle #1 \rangle}
\newcommand{\E}{\mathbb{E}}

\title{APMA 0360 Midterm 2 Review}
\author{Milan Capoor}
\date{12 April 2023}

\begin{document}
\maketitle
\section{Trig Identities}
\begin{enumerate}
    \item $\sin^2 x + \cos^2 x = 1$
    \item $1 + \tan^2 x = \sec^2 x$
    \item $\cos(-x) = \cos(x), \quad \sin(-x) = -\sin(x)$
    \item $\cos(2x) = \cos^2x - \sin^2 x$
    \item $\sin(2x) = 2\sin(x)\cos(x)$
    \item $\cos^2 x = \frac{1}{2} + \frac{1}{2}\cos(2x)$
    \item $\sin^2 x = \frac{1}{2} - \frac{1}{2}\cos(2x)$
\end{enumerate}

\section{Wave Equation}
\[u_{tt} = c^2 u_{xx}\]
\subsection{Factoring Method}
\begin{align*}
    u_{tt} - c^2 u_{xx} &= \left[\left(\frac{\partial}{\partial t}\right)^2 - c^2 \left(\frac{\partial}{\partial x}\right)^2\right]u\\
    &= \left(\frac{\partial}{\partial t} - c \frac{\partial}{\partial x}\right)\left(\frac{\partial}{\partial t} + c\frac{\partial}{\partial x}\right) = 0
\end{align*}
Let $v = \frac{\partial}{\partial t} + c\frac{\partial}{\partial x}$ si 
\[\left(\frac{\partial}{\partial t} - c\frac{\partial}{\partial x}\right)v = 0 \implies v_t - cv_x = 0\]
Solving the transport PDE,
\[v(x, t) = f(x + ct)\]
Solving for u,
\[u_t + cu_x = f(x + ct)\]
Homogeneous solution:
\[u_0(x, t) = G(x - ct)\]
Undetermined coeffiecients:
\[u_p = h(x + ct)\]
\begin{gather*}
    (h(x +ct))_t + c(h(x + ct))_x = f(x + ct)\\
    ch' + ch' = f\\
    h' = \frac{1}{2c}f\\
    h(x + ct) = \frac{1}{2c}F(x + ct)
\end{gather*}
so 
\[\boxed{u(x, t) = G(x - ct) + F(x + ct)}\]

\subsection{Coordinate Method}
Define orthogonal variables from the equation
\[\begin{cases}
    \xi = x - ct\\
    \eta = x + ct
\end{cases}\]
So 
\begin{align*}
    u_x &= u_\xi \, \xi_x + u_\eta \, \eta_x = u_\xi + u_\eta\\
    u_{xx} &= (u_x)_x = (u_x)_{\xi} \, \xi_x + (u_x)_\eta \, \eta_x\\
    &= u_{\xi \xi} + 2u_{\xi \eta} + u_{\eta \eta}\\
    u_{tt} &= c^2(u_{\xi \xi} - 2u_{\xi \eta} + u_{\eta \eta})
\end{align*}
Plug into the wave equation 
\begin{gather*}
    u_{tt} = c^2 u_{xx}\\
    c^2(u_{\xi \xi} - 2u_{\xi \eta} + u_{\eta \eta}) = c^2(u_{\xi \xi} + 2u_{\xi \eta} + u_{\eta \eta}) = 4u_{\xi \eta}\\
    u_{\xi \eta} = 0\\
    u_\xi = f(\xi)\\
    u = F(\xi) + G(\eta)
\end{gather*}
\[\boxed{u(x, t) = F(x - ct) + G(x + ct)}\]

\subsection{D'Alembert's Formula Derivation}
\[\begin{cases}
    u_{tt} = c^2 u_{xx}\\
    u(x, 0) = \phi(x)\\
    u_t(x, 0) = \psi(x)
\end{cases}\]

From above, the wave equation has general solution 
\[u(x, t) = F(x - ct) + G(x + ct)\]
so with initial conditions:
\[u(x, 0) = \phi(x) = F(x) + G(x)\]
\[u_t(x, 0)= \psi(x) = -cF'(x) + cG'(x) \implies -F'(x) + G'(x) = \frac{\psi(x)}{c}\]
Integrate over $[0, x]$:
\begin{gather*}
    \int_0^x -F'(s) + G'(s) \; ds = \int_0^x \frac{\psi(s)}{c}\; ds\\
    -F(x) + G(x) -(-F(0) + G(0)) = \frac{1}{c}\int_0^x \psi(s)\; ds
\end{gather*}
This gives the system of equations 
\[\begin{cases}
    -F(x) + G(x) = A + \frac{1}{c}\int_0^x \psi(s)\; ds\\
    F(x) + G(x) = \phi(x)
\end{cases}\]
\[\implies \begin{cases*}
    2G(x) = \phi(x) + A + \frac{1}{c}\int_0^x \psi(s)\; ds\\
    2F(x) = \phi(x) - A - \frac{1}{c}\int_0^xx \psi(s)\;ds
\end{cases*}\]
Solution:
\begin{align*}
    u(x, t) &= F(x - ct) + G(x+ct)\\
    &= \frac{1}{2}\phi(x - ct) - \frac{A}{2} - \frac{1}{2c}\int_0^{x - ct} \psi(s)\; ds + \frac{1}{2}\phi(x + ct) + \frac{A}{2} + \frac{1}{2c}\int_0^{x+ct}\psi(s)\;ds\\
    &= \frac{1}{2}(\phi(x -ct) + \psi(x + ct)) + \frac{1}{2c}\left(\int_{x-ct}^0 \psi(s)\;ds + \int_0^{x+ ct} \psi(s)\; ds\right)
\end{align*}
Which at last gives us d'Alembert's equation:
\[\boxed{u(x, t) = \frac{1}{2}(\phi(x -ct) + \psi(x + ct)) + \frac{1}{2c}\left(\int_{x-ct}^0 \psi(s)\;ds + \int_0^{x+ ct} \psi(s)\; ds\right)}\]

\section{Energy Methods}
\subsection{Wave Equation}
Steps:
\begin{enumerate}
    \item Multiply by a clever function (usually $u$ or $u_t$)
    \item Integrate with respect to $x$
\end{enumerate}

\textbf{Example:}
\[u_{tt} = c^2 u_{xx}\]
\[u_{tt}u_t = c^2 u_{xx} u_t\]
\[\int_{-\infty}^{\infty} u_{tt} u_t \; dx =\]
LHS by chain rule:
\[\int_{-\infty}^{\infty} u_{tt} u_t \; dx = \frac{d}{dt}\left(\frac{1}{2}\int_{-\infty}^{\infty} (u_t)^2 \; dx\right)\]

RHS by parts:
\begin{align*}
    c^2 \int_{-\infty}^{\infty} u_{xx} u_t \; dx &= [u_x u_t]_{-\infty}^\infty - \int_{-\infty}^{\infty} u_x u_{xt}\; dx\\
    &= -\int_{-\infty}^{\infty} u_x u_{xt} \; dx\\
    &= - \int_{-\infty}^{\infty} \frac{d}{dt}\left(\frac{1}{2}(u_x)^2\right)\; dx\\
    &= \frac{d}{dt}\left(-\frac{1}{2}\int_{-\infty}^{\infty} (u_x)^2\; dx\right)
\end{align*}
Then from $u_{tt} u_t = c^2u_{xx} u_t$,
\[\frac{d}{dt}\left(\frac{1}{2}\int_{-\infty}^{\infty} (u_t)^2 \; dx\right) = c^2 \frac{d}{dt}\left(-\frac{1}{2}\int_{-\infty}^{\infty} (u_x)^2\; dx\right)\]
\[\frac{d}{dt}\left(\frac{1}{2}\int_{-\infty}^{\infty} (u_t)^2 + c^2(u_x)^2 \; dx\right)= 0\]
\[\frac{d}{dt}E(t) = 0\]
Which proves that the energy function is constant. $\qed$

\subsection{Heat Equation}
Suppose u solves the PDE 
\[\begin{cases}
    u_t = Du_{xx}\\
    u(x, 0) = 0\\
    u(0, t) = 0\\
    u(l, t) = 0
\end{cases}\]
Then $u(x, t) = 0\quad \forall x,\, t$.

\textbf{Proof:}
Start with $u_t = Du_{xx}$ and multiply by $u$:
\[u_t u = Du_{xx} u\]
Integrate WRT $x$ on $[0, l]$ 
\[\int_0^l u_t u\; dx = D\int_0^l u_{xx}u \; dx\]
\[\frac{d}{dt} \left(\int_0^l u^2 \; dx\right) = D\left[u_x(l, t)u(l, t) - u_x(0, t) - \int_0^l u_x u_x \; dx\right]\] 
\[\frac{d}{dt} \left(\int_0^l u^2 \; dx\right) = -D\int_0^l (u_x)^2 \; dx\]
Define 
\[E(t) = \frac{1}{2}\int_0^l u^2 \; dx\]
and notice that $ -D\int_0^l (u_x)^2 \; dx \leq 0$ so 
\[\frac{d}{dt}E(t) \leq 0\] 
which means that $E(t) \leq E(0)$ and 
\[E(t) = \frac{1}{2}\int_0^l (u(x, t))^2 \; dx \leq E(0) = \frac{1}{2}\int_0^l(u(x, 0))^2\; dx = 0\]
which means that $0 \leq E(t) \leq E(0) = 0 \implies E(t)= 0$ so $u(x, t) = 0$ for all $x$ and $t. \qed$

\subsection{Uniqueness}
\textbf{Fact:} there is at most one solution of 
\[\begin{cases*}
    u_{tt} = c^2 u_{xx}\\
    u(x, 0) = \phi(x)\\
    u_t(x, 0) = \psi(x)
\end{cases*}\]

\textbf{Proof:} Suppose there are two functions $u$ and $v$ that solve the above PDE. Then, checking if $w$ solves the PDE too,
\begin{gather*}
    w_{tt} = c^2 w_{xx}\\
    w(x, 0) = u(x, 0) - v(x, 0) = \phi(x) - \phi(x) = 0\\
    w_t(x, 0) = u_t(x, 0) - v_t(x, 0) = \psi(x) - \psi(x) = 0
\end{gather*}
Using the energy method as above, we show that 
\[E(t) = \frac{1}{2}\int_{-\infty}^{\infty} (w_t)^2 + c^2(w_x)^2\; dx\]
is constant so $E(t) = E(0)$ and 
\[\frac{1}{2}\int_{-\infty}^{\infty} (w_t)^2 + c^2(w_x)^2\; dx = \frac{1}{2}\int_{-\infty}^{\infty} (w_t(x, 0))^2 + c^2(w_x(x, 0))^2 \; dx\]
by the initial conditions $w_t(x, 0) = 0$ and 
\[w(x, 0) = 0 \implies (w(x, 0))_x = 0_x \implies w_x(x, 0) = 0\]
so the above becomes 
\[\frac{1}{2}\int_{-\infty}^{\infty} (w_t)^2 + c^2(w_x)^2 \; dx =0\]
Then as $(w_t)^2 + c^2(w_x)^2 \geq 0$, and the integral is positive, 
\[(w_t)^2 + c^2(w_x)^2 = 0\]
so 
\[\begin{cases}
    w_t = 0\\
    w_x = 0
\end{cases} \implies w(x, t) = C\]
but as $w(x, 0) = 0$, we know that 
\[w(x, t) = 0 \implies u - v= 0 \implies u = v\]
so there is at most one solution. $\qed$

\section{Separation of Variables}
\subsection{Wave}
\textbf{Example:} 
\[\begin{cases}
    u_{tt} = c^2u_{xx}\\
    u(0, t) = 0\\
    u(1, t) = 0\\
    u(x, 0) = x^2\\
    u_t(x, 0) = e^x
\end{cases}\]

\textbf{Solution:}
Assume that $u(x, t) = X(x)T(t)$ so 
\begin{gather*}
    XT'' = c^2X''T\\
    \frac{T''}{c^2T} = \frac{X''}{X} = \lambda
\end{gather*}

Initial conditions:
\[\begin{cases*}
    X'' = \lambda X\\
    u(0, t) = 0 \implies X(0) = 0\\
    u(1, t) = 0 \implies X(1) = 0
\end{cases*}\]

\emph{Boundary Value Problem:}
$\lambda > 0$:
\[X = Ae^{\omega x} + Be^{-\omega x}\]
\[X(0) = A + B = 0 \implies X = Ae^{\omega x} - Ae^{\omega x}\]
\[X(1) = Ae^{\omega} - Ae^{-\omega} = 0 \implies \omega = -\omega \implies \omega = 0\]

$\lambda = 0$:
\[X(x) = A + Bx\]
\[X(0) = A = 0 \implies X(x) = Bx\]
\[X(1) = B = 0 \implies X(x) = 0\]

$\lambda <0$:
\[X(x) = A\cos(\omega x) + B\sin(\omega x)\]
\[X(0) = A = 0 \implies X(x) = B\sin(\omega x)\]
\[X(1) = 0 \implies \sin(\omega) = 0 \implies \omega = \pi m \quad (m = 1, 2, ...)\]
So $X(x) = \sin(\pi mx)$ corresponding to $\lambda = -(\pi m)^2 \quad (m =1, 2, ...)$

\emph{Conclusion}
\[\frac{T''}{c^2T} = \lambda = -(\pi m)^2\]
\[T'' + (\pi m c)^2 T = 0\]
\[T(t) = A\cos(\pi mct) + b\sin(\pi mct)\]
So via linearity and the definition of X and T, 
\[u(x, t) = \sum_{m=1}^\infty (A_m\cos(\pi mct) + B_m\sin(\pi mct))\sin(\pi mx)\]

\emph{Initial conditions:}
\[u(x, 0) = \boxed{\sum_{m=1}^\infty A_m\sin(\pi mx) = x^2}\]
\[u_t(x, t) = \sum_{m=1}^\infty (-A_m\pi mc \sin(\pi mct) + B_m \pi mc \cos(\pi mct))\cos(\pi mx)\]
\[u_t(x, 0) = \boxed{\sum_{m=1}^\infty B_m\pi mc \sin(\pi mx) = e^x}\]

\subsection{Heat}
\textbf{Example:}
\[\begin{cases}
    u_t = Du_{xx}\\
    u_x(0, t) = 0\\
    u_x(\pi, t) = 0\\
    u(x, 0) = x^2
\end{cases}\]

\textbf{Solution:}
Suppose $u(x, t) = X(x) T(t)$. Then 
\begin{gather*}
    XT' = DX''T\\
    \frac{X''}{X} = \frac{T'}{DT} = \lambda
\end{gather*}

\[\begin{cases*}
    X'' = \lambda X\\
    u_x(0, t) = 0\implies X'(x) = 0\\
    u_x(\pi, t) = 0 \implies X'(\pi) = 0
\end{cases*}\]

\emph{Boundary Value Problem:}
$\lambda > 0$: 
\[X = Ae^{\omega x} + Be^{-\omega x}\]
\[X'(0) = A\omega - B\omega = 0 \implies A = B\]
\[X'(\pi) = A\omega e^{\pi \omega} - A\omega e^{\pi \omega} = 0 \implies \omega = 0\]

$\lambda = 0$:
\[X = A + Bx\]
\[X'(0) = B =0 \implies X = A\]
So $\lambda = 0$ is an eigenvalue with eigenfunction $X(x) =A$

$\lambda < 0$:
\[X(x) = A\cos(\omega x) + B\sin(\omega x)\]
\[X'(0) = -A\omega \sin(0) + B\omega \cos(0) = 0 \implies B =0\]
\[X(x) = A\cos(\omega x)\]
\[X'(\pi) = 0 \implies \sin(\omega \pi) = 0\]
So $\lambda = -m^2$ are the eigenvalues corresponding to eigenfunction $X(x) = \cos(mx)$

\emph{T equation}
\[\frac{T'}{DT} = \lambda = -m^2 \implies T' = -m^2DT\]
\[T(t) = e^{-m^2Dt}\]
so 
\[u(x, t) = X(x)T(t) = e^{-m^2Dt}\cos(mx) \quad (m= 0, 1,2, ...)\]

\emph{Initial Conditions}
By linearity,
\[u(x, t) = \sum_{m=0}^\infty A_me^{-m^2Dt}\cos(mx)\]
\[u(x, 0) = \boxed{x^2 = \sum_{m=0}^\infty A_m \cos(mx)}\]

\subsection{Laplace}
\emph{Note:} For Laplace's equation, you don't always start with the X equation, sometimes you start with Y . Always choose the variable that gives you a 0 boundary condition

\textbf{Example:}
\[\begin{cases}
    u_{xx} + u_{yy} = 0\\
    u(0, y) = 0\\
    u(\pi, y) = 0\\
    u(x, 0) = x\\
    u(x, 1) = 3
\end{cases}\]

\textbf{Solution:}
Assume $u(x, y) = X(x)Y(y)$. 
\begin{gather*}
    X''Y + XY'' = 0\\
    \frac{X''}{X} = -\frac{Y''}{Y} = \lambda
\end{gather*}

\emph{Initial conditions:}
\begin{gather*}
    u(0, y) = 0 \implies X(0) = 0\\
    u(\pi, y) = 0 \implies X(\pi) = 0
\end{gather*}

\emph{Boundary Value:}
\[X'' = \lambda X\]
$\lambda > 0$: 
\[X = Ae^{\omega x} + Be^{\omega x}\]
\[X(0) = A + B = 0 \implies A = -B\]
\[X(\pi) = Ae^{\omega \pi} - Ae^{-\omega \pi} = 0 \implies \omega = 0\]

$\lambda = 0$:
\[X = A + Bx\]
\[X(0) = A = 0\]
\[X(\pi) = B\pi = 0 \implies B =0\]

$\lambda < 0$:
\[X(x) = A\cos(\omega x) + B\sin(\omega x)\]
\[X(0) = A = 0 \implies X(x) = B\sin(\omega x)\]
\[X(\pi) = B\sin(\pi \omega) = 0 \implies \sin(\pi m) = 0 \quad (m=1, 2, ...)\]
So $X(x) = \sin(mx)$ corresponding to $\lambda = -m^2$

\emph{Back to Laplace:}
\[Y''(y) = -\lambda Y(y) = m^2Y(y)\]
\begin{align*}
    Y(y) &= Ae^{my} + Be^{-my}\\
    &= A(\cosh(my) + \sinh(my)) + B(\cosh(my)- \sinh(my))\\
    &= (A + B)\cosh(my) + (A-B)\sinh(my)\\
    &= A\cosh(my) + B\sinh(my)
\end{align*}
\[u(x, y) = X(x)Y(y) = (A\cosh(my) + B\sinh(my))\sin(mx)\]
\[u(x, y) = \sum_{m=1}^\infty (A_m\cosh(my) + B_m\sinh(my))\sin(mx)\]
\[u(x, 0) = \boxed{\sum_{m=1}^\infty A_m\sin(mx) = x}\]
\[u(x, 1) = \boxed{\sum_{m=1}^\infty (A_m\cosh(m) + B_m\sinh(m))\sin(mx) = 3}\]

\section{Fourier Series}
\subsection{Sine series}
Because $\{\sin(mx)\; | \;m= 1, 2, ...\}$ is orthogonal, for
\[f(x) = \sum_{m=1}^\infty A_m\sin(mx)\]
on $(0, \pi)$ we have
\[A_m = \frac{f \cdot \sin(mx)}{\sin(mx) \cdot \sin(mx)} = \frac{\int_0^\pi f(x)\sin(mx)\; dx}{\int_0^\pi \sin^2(mx)\; dx} = \frac{2}{\pi}\int_0^\pi f(x)\, \sin(mx)\; dx.\]

More generally for 
\[f(x) = \sum_{m=1}^\infty A_m \sin(\frac{\pi mx}{L})\]
on $(0, L)$ we have 
\[\boxed{A_m = \frac{2}{L}\int_0^L f(x)\, \sin(\frac{\pi mx}{L})\; dx}\]
\subsection{Cosine series}
Similarly, for 
\[f(x) = \sum_{m=0}^\infty A_m\cos(mx)\]
on $(0, \pi)$, 
\[A_m = \frac{f\cdot \cos(mx)}{\cos(mx) \cdot \cos(mx)} = \frac{2}{\pi}\int_0^\pi f(x)\cos(mx) \; dx\]
but 
\[A_0 = \frac{1}{\pi}\int_0^\pi f(x)\; dx.\]

Generally, for 
\[f(x) = \sum_{m=0}^\infty A_m \cos(\frac{\pi mx}{L}) \quad 0 < x < L\]
we have 
\begin{empheq}[box=\fbox]{align*}
    A_m &= \frac{2}{L}\int_0^L f(x)\cos(\frac{\pi mx}{L})\; dx\\
    A_0 &= \frac{1}{L}\int_0^L f(x)\;dx
\end{empheq}
\subsection{Full series}
When the interval is extended to $(-\pi, \pi)$ we redefine the dot product as 
\[f \cdot g = \int_{-\pi}^\pi f(x)\, g(x)\; dx\]
so for 
\[f(x) = \sum_{m=0}^\infty A_m\cos(\frac{\pi mx}{L}) + B_m \sin(\frac{\pi mx}{L})\] 
on $(-\pi, \pi)$ we have 
\begin{empheq}[box=\fbox]{align*}
    A_m &= \frac{1}{L}\int_{-L}^L f(x)\cos(\frac{\pi mx}{L})\; dx\\
    B_m &= \frac{1}{L}\int_{-L}^L f(x)\sin(\frac{\pi mx}{L})\; dx\\
    A_0 &= \frac{1}{2L}\int_{-L}^L f(x)\; dx\\
    B_0 &= 0
\end{empheq}
\subsection{Complex series}
For complex numbers, we again redefine the dot product such that 
\[f \cdot g = \int_{-\pi}^\pi f(x) \overline{g(x)}\]
where 
\[\overline{a +bi} = a -bi\]
so on $-\pi < x < \pi$,
\[f(x) = \sum_{m=-\infty}^\infty C_m e^{imx}\]
we have 
\[C_m = \frac{f \cdot e^{imx}}{e^{imx} \cdot e^{imx}} = \frac{1}{2\pi} \int_{-\pi}^\pi f(x)e^{-imx}\; dx.\]

Generally, with 
\[f(x) = \sum_{m=-\infty}^\infty C_me^{i\left(\frac{\pi mx}{L}\right)}\]
on $(-L, L)$:
\[\boxed{C_m = \frac{1}{2L} \int_{-L}^L f(x)e^{-i\left(\frac{\pi mx}{L}\right)}\; dx}\]

\subsection{Parseval's Identity}
\textbf{Definition:} $||u|| = \sqrt{u \cdot u}$ and $||cu|| = |c| \; ||u||$

\textbf{Pythagorean Theorem:} If $\{u, v, w\}$ is orthogonal,
\[||u + v + w||^2 = ||u||^2 + ||v||^2 + ||w||^2\]

Then on $(0, \pi)$, because $\{\sin(mx)\}$ is orthogonal,
\begin{align*}
    f(x) &= \sum_{m=1}^\infty A_m \sin(mx)\\
    ||f||^2 &= \left|\left|\sum_{m=1}^\infty A_m \sin(mx)\right|\right|^2\\
    &= \sum_{m=1}^\infty ||A_m \sin(mx)||^2\\
    &= \sum_{m=1}^\infty |A_m|^2 ||\sin(mx)||^2\\
    \int_0^\pi (f(x))^2\; dx &= \sum_{m=1}^\infty |A_m|^2 \int_0^\pi \sin^2(mx) \; dx\\
    &= \frac{\pi}{2}\sum_{m=1}^\infty |A_m|^2
\end{align*}
so 
\[\boxed{\sum_{m=1}^\infty |A_m|^2 = \frac{2}{\pi}\int_0^\pi (f(x))^2\; dx}\]

\section{Laplace Equation}
\subsection{Derivation}
From the 2D heat equation where $u = u(x, y, t)$
\[u_t = D(u_{xx} + u_{yy})\]
We assume that $\lim_{t\to \infty} u_t = 0$ so 
\[0 = D(u_{xx} + u_{yy}) \implies u_{xx} + u_{yy} = 0\]
\subsection{Rotational Invariance}
\textbf{Theorem:} 

Let $\Delta u(x, y) = 0$. Then for some constant $\theta$ where 
\[\begin{bmatrix}
    x'\\y'
\end{bmatrix} = \begin{bmatrix}
    \cos \theta & -\sin \theta\\
    \sin \theta & \cos \theta
\end{bmatrix} \begin{bmatrix}
    x\\y
\end{bmatrix}\]
\[u_{x' x'} + u_{y' y'} = u_{xx} + u_{yy} = 0\]

\textbf{Proof:}
\[\begin{cases}
    x' = \cos(\theta) x - \sin(\theta)y\\
    y' = \sin(\theta)x + \cos(\theta)y
\end{cases}\]
\begin{align*}
    u_x &= u_{x'} \cdot x'_x + u_{y'} \cdot y_x' = (u_{x'})\cos(\theta) + (u_{y'})\sin(\theta)\\
    u_{xx} &= (u_x)_{x'} \cdot x'_x + (u_x)_{y'} \cdot y'_x = u_{x'x'}\cos^2(\theta) + 2u_{x'y'}\sin(\theta)\cos(\theta) + u_{y'y'} \sin^2(\theta)\\
    u_y &= u_{x'} \cdot x'_y + u_{y'} \cdot y'_y = -u_{x'}\sin(\theta) + u_{y'}\cos(\theta)\\
    u_{yy} &= (u_y)_{x'} \cdot x'_y + (u_y)_{y'} \cdot y'_y = u_{x'x'}\sin^2(\theta) - 2u_{y'x'}\cos(\theta)\sin(\theta) + u_{y'y'}\cos^2(\theta)
\end{align*}
\[u_{xx} + u_{yy}= u_{x'x'} + u_{y'y'} \qed\]

\subsection{Fundamental Solution}
\textbf{Example:}
Use the Polar Laplace 
\[u_{rr} + \frac{1}{r}u_r =0\]
with constants $A = -\frac{1}{2\pi}$ and $B =0$ to derive the fundamental solution to the Laplace equation. 

\textbf{Solution:}
\emph{Integrating factors:}
\[e^{\int \frac{1}{r}\; dr} = e^{\ln r} = r\]
\[ru_{rr} + u_r = 0 \implies (ru_r)_r = 0\]
\[ru_{r} = A \implies u_r = \frac{A}{r}\]
\[u = A \ln r + B\]
\[u(x, y) = A\ln(\sqrt{x^2 + y^2}) + B\]
\[\boxed{\Phi(x, y) = -\frac{1}{2\pi}\ln(\sqrt{x^2 + y^2})}\]
\end{document}