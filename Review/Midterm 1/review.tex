\documentclass[12pt]{article} 
\usepackage[utf8]{inputenc}
\usepackage{geometry}
\geometry{letterpaper}
\usepackage{graphicx} 
\usepackage{parskip}
\usepackage{booktabs}
\usepackage{array} 
\usepackage{paralist} 
\usepackage{verbatim}
\usepackage{subfig}
\usepackage{fancyhdr}
\usepackage{sectsty}

\pagestyle{fancy}
\renewcommand{\headrulewidth}{0pt} 
\lhead{}\chead{}\rhead{}
\lfoot{}\cfoot{\thepage}\rfoot{}

%%% SECTION TITLE APPEARANCE
\allsectionsfont{\sffamily\mdseries\upshape} 

%%% ToC (table of contents) APPEARANCE
\usepackage[nottoc,notlof,notlot]{tocbibind} 
\usepackage[titles,subfigure]{tocloft}
\renewcommand{\cftsecfont}{\rmfamily\mdseries\upshape}
\renewcommand{\cftsecpagefont}{\rmfamily\mdseries\upshape} %

\usepackage{amsmath}
\usepackage{amssymb}
\usepackage{empheq}
\usepackage{xcolor}
\renewcommand{\L}[1]{\mathcal{L}\{#1\}}
\newcommand{\ans}[1]{\boxed{\text{#1}}}
\newcommand{\vecs}[1]{\langle #1\rangle}
\renewcommand{\hat}[1]{\widehat{#1}}
\newcommand{\F}[1]{\mathcal{F}(#1)}
\renewcommand{\P}{\mathbb{P}}
\newcommand{\R}{\mathbb{R}}
\newcommand{\qed}{\quad \blacksquare}
\newcommand{\brak}[1]{\langle #1 \rangle}

\title{APMA 0360: Midterm 1 Review Sheet}
\author{Milan Capoor}
\date{1 March 2023}
\begin{document}
\maketitle
\section*{Basic Math}
\textbf{Useful Trig:}
\begin{itemize}
    \item $\sin^2 x + \cos^2 x = 1$
    \item $1 + \tan^2 x = \sec^2 x$
    \item $\cos(-x) = \cos x$
    \item $\sin(-x) = -\sin x$
    \item $\cos(2x) = \cos^2(x) - \sin^2(x)$
    \item $\sin(2x) = 2\sin x \cos x $
    \item $\cos^2 x = \frac{1}{2} + \frac{1}{2} \cos (2x)$
    \item $\sin^2 x = \frac{1}{2} - \frac{1}{2}\cos (2x)$
    \item $\int \tan x \; dx = \ln | \sec x | + C$
    \item $\int \sec x\; dx = \ln|\sec x + \tan x| + C$
    \item $\int \frac{1}{x^2 + 1}\; dx = \tan^{-1} x + C$
    \item $\int \sin^2 x\; dx = \frac{x}{2} - \frac{1}{4}\sin(2x)$   
    \item $\int \cos^2 x \; dx = \frac{x}{2} + \frac{1}{4} \sin(2x)$ 
\end{itemize}

\section*{Introduction}
\textbf{Check if a function solves the PDE:} plug the function in (differentiate) and see if you get an identity

\textbf{Solve a simple PDE:} Use ODEs
Examples:
\begin{itemize}
    \item $u_x = 0 \implies u(x, y) = f(y)$
    \item $u_{xx} = 0 \implies u_x = f(y) \implies u(x, y) = xf(y) + g(y)$
    \item $u_{xx} +u = 0 \overset{y'' + y = 0}{\Longrightarrow} u(x, y) = A(y) \cos x + B(y) \sin x$
    \item $u_{xy}= 0 \implies (u_x)_y = 0 \implies u_x = f(x) \implies u(x, y) = F(X) + G(y)$
\end{itemize}

\textbf{Classification:}

\emph{Order:} highest derivative 

\emph{Constant coefficient}

\emph{Linear:} coefficients depend on x, y but not u 
\begin{itemize}
    \item $L(u + v) = L(u) + L(v)$
    \item $L(cu) = cL(u)$
\end{itemize}

\emph{Homogeneous:} RHS is 0 

\emph{Elliptic forms of Second Order PDEs:}
\[au_{xx} + bu_{xy} + cu_{yy} + du_x + eu_y + fu = g(x,y)\]
Let $D = b^2 - 4ac$, 
\begin{itemize}
    \item elliptic if $D < 0$
    \item parabolic if $D = 0$
    \item hyperbolic if $D > 0$
\end{itemize}

\section*{First Order Linear}
\textbf{Directional Derivative:}
\begin{enumerate}
    \item Write as a directional derivative
    \item Show that u is constant on characteristic lines 
\end{enumerate}

Example 1: $au_x + bu_y = 0$
\[\brak{u_x, u_y} \cdot \brak{a, b} = \nabla u \cdot \vec{v} = 0\]
\[m = \frac{b}{a} \implies y = \frac{b}{a}x + C \implies ay - bx = C\]
\[u(x, y) = f(ay - bx)\]

Example 2: $u_x + yu_y = 0$
\[\nabla u \cdot (1, y) = 0\]
\[\frac{y}{1} = y' \quad \text{slope = derivative} \]
\[y' = y \implies y = Ce^x \implies ye^{-x} = C \implies u(x, y) = f(ye^{-x})\]

\textbf{Coordinate Method:}
\begin{enumerate}
    \item Define new variables $\xi$ and $\eta$ that are perpendicular
    \item Rewrite the PDE in terms of $\xi$ and $\eta$ with the chain rule
    \item Substitute definitions 
    \item Solve
\end{enumerate}

Example: $2u_x + 3u_y = 0$
\[\begin{cases}
    \xi = 2x  +3y \quad \text{(from equation)}\\
    \eta = -3x + 2y \quad \text{(perpendicular)}
\end{cases}\]
\begin{align*}
    \frac{\partial u}{\partial x} = \frac{\partial u}{\partial \xi}\frac{\partial \xi}{\partial x} + \frac{\partial u}{\partial \eta}\frac{\partial \eta}{\partial x} = 2u_\xi - 3u_\eta\\
    \frac{\partial u}{\partial y} = \frac{\partial u}{\partial \xi}\frac{\partial \xi}{\partial y} + \frac{\partial u}{\partial \eta}\frac{\partial \eta}{\partial y} = 3u_\xi + 2u_\eta\\
\end{align*}
\begin{align*}
    2u_x + 3u_y &= 2(2u_\xi - 3u_\eta) + 3(3u_\xi + 2u_\eta)\\
    &= 4u_\xi - 6u_\eta + 9u_\xi + 6u_\eta = 0\\
    &= 13u_\xi = 0
\end{align*}
\[\implies u_\xi = 0 \implies u = f(\eta) \implies u(x, y) = f(2y - 3x)\]

\textbf{Transform Method:} Rewrite the derivatives of u in terms of a new PDE v and solve 

\section*{The Transport Equation}
\textbf{Derivation}
The mass on an interval is 
\[M = \int_0^b u(x, t)\; dx\]
But as mass is conserved between time intervals, 
\[M_1 = M_2 = \int_0^b u(x, t)\; dx = \int_{ch}^{b + ch}u(x, t + h)\; dx\]
where c is the speed of the fluid. Then,
\[\frac{d}{db} \int_0^b u(x, t)\; dx = \frac{d}{db} \int_{ch}^{b+ch} u(x, t + h)\; dx\]
By FTC, 
\[u(b, t) = u(b + ch, t + h)\]
Differentiate WRT h, 
\[0 = u_x \cdot (b+ch)_h + u_t \cdot (t+h)_h = cu_x + u_t\]

\textbf{Solution}
\[u_t + cu_x = 0 \implies cu_x - u_t = 0 \implies u(x, t) = f(x -ct)\]

\section*{The Fourier Transform}
\textbf{The Gaussian:}
\[I = \int_{-\infty}^{\infty} e^{-x^2}\; dx\]
\begin{align*}
    I^2 &= (I)(I)\\
    &= \left(\int_{-\infty}^{\infty} e^{-x^2}\; dx\right)\left(\int_{-\infty}^{\infty} e^{-y^2}\; dy\right) \quad \text{(trick)}\\
    &= \int_{-\infty}^{\infty} \int_{-\infty}^{\infty} e^{-x^2} e^{-y^2}\; dx \, dy\\
    &= \int_{-\infty}^{\infty} \int_{-\infty}^{\infty} e^{-(x^2 + y^2)} \; dx \, dy\\
    &= 2\pi \int_{0}^{\infty} re^{-r^2}\; dr\\
    &= \pi
\end{align*}
\[\implies I = \int_{-\infty}^{\infty} e^{-x^2}\; dx = \sqrt{\pi}\]

\emph{Application to sin and cos:}
\begin{align*}
    I &= \int_{-\infty}^{\infty} e^{ix^2}\; dx\\
    I^2 &= \int_{-\infty}^{\infty} \int_{-\infty}^{\infty} e^{i(x^2 + y^2)}\; dy\; dx\\
    &= \int_0^{2\pi} \int_0^\infty re^{ir^2}\; dr d\theta\\
    &= \int_0^{2\pi} \left[\frac{e^{ir^2}}{2i}\right]_{r=0}^{r=\infty} \; d\theta = \int_0^{2\pi} -\frac{1}{2i} \; d\theta\\
    &= \pi i\\
    I &= \int_{-\infty}^{\infty} e^{ix^2}\; dx = \sqrt{\pi i}
\end{align*}

\begin{align*}
    \sqrt{\pi i} &= \sqrt{\pi} \sqrt{i}\\
    &= \sqrt{\pi} \sqrt{e^{i \frac{\pi}{2}}}\\
    &= \sqrt{\pi} e^{i \frac{\pi}{4}}\\
    &= \sqrt{\frac{\pi}{2}} + i\sqrt{\frac{\pi}{2}}
\end{align*}
\[\int_{-\infty}^{\infty} e^{ix^2}\; dx = \int_{-\infty}^{\infty} \cos(x^2) + i\sin(x^2) \; dx = \sqrt{\frac{\pi}{2}} + i\sqrt{\frac{\pi}{2}}\]
\[\begin{cases}
    \int_{-\infty}^{\infty} \cos(x^2)\; dx = \sqrt{\frac{\pi}{2}}\\
    \int_{-\infty}^{\infty} \sin(x^2)\; dx = \sqrt{\frac{\pi}{2}}
\end{cases}\]

\textbf{The Fourier Transform:}
\[\hat{f}(\kappa) = \int_{-\infty}^{\infty} f(x)e^{i\kappa x}\; dx\]

Example: $\F{e^{-x^2}}$
\begin{align*}
    \hat{f}(\kappa) &= \int_{-\infty}^{\infty} e^{-x^2}e^{i\kappa x}\; dx\\
    \hat{f}'(\kappa) &= i\int_{-\infty}^{\infty} xe^{-x^2} e^{i\kappa x}\; dx\\
    &\overset{\text{IBP}}{=} i\left[-\frac{1}{2}e^{-x^2}e^{i\kappa x}\right]_{-\infty}^\infty - i \int_{-\infty}^{\infty} -\frac{1}{2} e^{-x^2}e^{i\kappa x}\; dx\\
    &= -\frac{\kappa}{2} \hat{f}(\kappa)
\end{align*}
\[\hat{f}'(\kappa)= -\frac{\kappa}{2} \hat{f}(\kappa) \implies \hat{f}(\kappa) =Ce^{-\kappa^2/4} \implies \hat{f}(\kappa) = \sqrt{\pi}e^{-\kappa^2/4}\]

Generally,
\[\F{e^{-ax^2}} = \sqrt{\frac{\pi}{a}} e^{-\frac{\kappa^2}{4a}}\]

\textbf{Derivatives:}
\[\hat{f}'(\kappa) = (-i\kappa) \hat{f}(\kappa)\]

Proof:
\begin{align*}
    \hat{f}'(\kappa) &= \int_{-\infty}^{\infty} f'(x) e^{i\kappa x}\; dx\\
    &\overset{\text{IBP}}{=} \left[f(x)e^{i\kappa x}\right]_{-\infty}^\infty - \int_{-\infty}^{\infty} f(x) \frac{d}{dx}e^{i\kappa x}\; dx\\
    &= 0 - i\kappa \int_{-\infty}^{\infty} f(x) e^{i\kappa x}\; dx\\
    &= -i\kappa \hat{f}(\kappa)
\end{align*}

\textbf{Convolution:}
\[(f \star g)(x) = \int_{-\infty}^{\infty} f(x - y)g(y)\; dy\]
\[\F{(f \star g)(\kappa)} = \hat{f}(\kappa) \cdot \hat{g}(\kappa)\]

\textbf{Inverse Fourier:}
\[\mathcal{F}^{-1}(x) = \frac{1}{2\pi} \int_{-\infty}^{\infty} f(\kappa) e^{-i\kappa x}\; d\kappa\]

\textbf{Shifting:}
\[\F{f(x - a)} = e^{i\kappa a} \hat{f}(\kappa)\]

\section*{The Heat Equation}
\textbf{Derivation}
Let $u= u(x,t)$ be the concentration of particles that move only left or right along a rod. 
Focus on the change in the concentration along an interval of length $h$ around $(x, t)$.
Assume that as $t \to t + \tau$, each particle moves left or right with equal property, 
\[hu(x, t + \tau) = hu(x, t) + \Delta u(x, t)\]
\[\Delta u(x, t) = \text{in - out} = \frac{1}{2}hu(x - h, t) + \frac{1}{2}hu(x + h, t) - hu(x, t)\]
(interpretation: all the original particles at $(x, t)$ leave but half from $(x - h, t)$ and half from $(x + h, t)$ come)
Therefore, 
\[hu(x, t + \tau) - hu(x, t)= \frac{1}{2}hu(x - h, t) + \frac{1}{2}hu(x + h, t) - hu(x, t) \]

Make some transformations and write in the limit form of the derivative to get 
\[\frac{u(x, t + \tau) - u(x, t)}{\tau} = \frac{h^2}{2\tau}\left(\frac{u(x - h, t) - 2u(x, t) + u(x + h, t)}{h^2}\right)\]
\[u_t = \frac{h^2}{2\tau} u_{xx}\]
Then define tau such that $h^2/2\tau = D$ giving 
\[u_t = Du_{xx}\]

\textbf{Solving}
\[\begin{cases}
    u_t = Du_{xx}\\
    u(x, 0) = f(x)
\end{cases}\]

Solution:
\begin{align*}
    \F{u_t} =\F{Du_{xx}}\\
    \frac{d}{dt} \hat{u} &= D(-i\kappa)^2 \hat{u}\\
    &= -D\kappa^2 \hat{u}\\
    \hat{u} &= \hat{u}(\kappa, 0)e^{-D\kappa^2t}\\
    &= \hat{f}(\kappa) e^{-D\kappa^2t}
\end{align*}
From the Gaussian, 
\[e^{-\frac{\kappa^2}{4a} = e^{-D\kappa^2t}}\implies a = \frac{1}{4Dt} \implies \sqrt{\frac{a}{\pi}} = \frac{1}{\sqrt{4\pi Dt}}\]
\[e^{-\kappa^2Dt} = \F{\frac{1}{\sqrt{4\pi Dt}}e^{-\frac{x^2}{4Dt}}}\]
\[\hat{u} = \hat{f}(\kappa) \cdot \F{\frac{1}{\sqrt{4\pi Dt}}e^{-\frac{x^2}{4Dt}}} = \F{(f \star g)(\kappa, t)}\]
\[u(x, t) = \int_{-\infty}^{\infty} f(y) g(x - y, t)\; dy\]
\[u(x, t) = \frac{1}{\sqrt{4\pi Dt}} \int_{-\infty}^{\infty} f(y) e^{-\frac{(x - y)^2}{4Dt}} \; dy\]

If there is a specified exponential initial condition, simplify the exponent, complete the square, and u-sub to form the gaussian, then eliminate with $\sqrt{\pi}$

\textbf{Properties}
\begin{enumerate}
    \item ``Infinite speed of propagation:" If $f \geq 0$ is positive somewhere and continuous, u is positive everywhere
    \emph{Proof:} the exponential function is positive and if f is continuous, it is positive around a region $x_0$ so the full infinite integrand is positive (so $\geq$ to a subset) but that subset is strictly positive so u is strictly positive 
    \item ``Smoothness:'' u is infinitely differentiable 
    \emph{Proof:} the exponential term is infinitely differentiable
    \item ``Irreversibility:'' $u(x, 0)$ cannot be determined from $u(x, 1)$
    \emph{Proof:} $u(x, 1) = |x|$ can be given but this is not smooth so violates the earlier property 
    \item ``Heat dissipates over time:'' 
    \[\lim_{t \to \infty}u(x, t) = 0\]
    \emph{Proof:}
    \begin{align*}
        |u(x, t)| &= \left|\frac{1}{\sqrt4\pi D t} \int_{-\infty}^\infty e^{-\frac{(x - y)^2}{4Dt}} f(y) \; dy\right|\\
        &\leq \frac{1}{\sqrt4\pi D t} \int_{-\infty}^\infty |f(y)| \underbrace{e^{-\frac{(x - y)^2}{4Dt}}}_{\leq 1} \; dy\\
        &\leq \frac{1}{\sqrt4\pi D t} \int_{-\infty}^\infty |f(y)| \; dy\\
        &= C \overset{t \to \infty}{\longrightarrow} 0
    \end{align*}
    \item ``Boundedness:'' if $|f(x)| \leq M$, then $|u(x, t)| \leq M$
    \emph{Proof:}
    \[|u(x, t)| \leq \frac{1}{\sqrt{4\pi Dt}} \int_{-\infty}^{\infty} |f(y)| e^{-\frac{(x - y)^2}{4Dt}} \; dy \leq M\]
    \item ``Conservation of Mass:'' 
    \[\int_{-\infty}^{\infty} u(x, t)\; dx = \int_{-\infty}^{\infty} f(x) \; dx\]
    \emph{Proof:}
    \[\frac{d}{dt}\int_{-\infty}^{\infty} u(x, t)\; dx = 0\] 
    \[\implies \int_{-\infty}^{\infty} u(x, t)\; dx = \int_{-\infty}^{\infty} u(x, 0)\; dx = \int_{-\infty}^{\infty} f(x)\; dx\]
\end{enumerate}
\section*{The Wave Equation}
\[u_{tt} = c^2 u_{xx}\]
\textbf{Factoring Method}
Apply the differential operator to the difference:
\[u_{tt} - c^2 u_{xx} = \left(\frac{\partial}{\partial t} - c \frac{\partial}{\partial x}\right)\left(\frac{\partial}{\partial t} + c\frac{\partial}{\partial x}\right) = 0\]
Let $v = (\frac{\partial}{\partial t} + c\frac{\partial}{\partial x})$ so 
\[\left(\frac{\partial}{\partial t} - c \frac{\partial}{\partial x}\right)v = 0 \implies v_t - cv_x = 0 \implies v(x, t) = f(x + ct)\]
Solve for u:
\[v = \left(\frac{\partial}{\partial t} + c\frac{\partial}{\partial x}\right)u = u_t + cu_x \implies u_t + cu_x = f(x + ct)\]
Solve the inhomogeneous transport equation using undetermined coefficients:
\[u_p = h(x + ct) \overset{\text{plug in}}{\Longrightarrow} h(x + ct) = \frac{1}{2c}F(x + ct)\]
where $F$ is an antiderivative of $f$ giving a general solution
\[u(x, t) = F(x + ct) + G(x - ct)\]

\textbf{Coordinate Method}
Define variables 
\[\begin{cases}
    \xi = x - ct\\
    \eta = x + ct
\end{cases}\]
Chain rule:
\[\begin{cases}
    u_x = u_\xi \xi_x + u_\eta \eta_x = u_\xi + u_\eta\\
    u_{xx} = (u_x)_\xi \xi_x + (u_x)_\eta \eta_x  = (u_\xi)_\xi + (u_\eta)_\eta = u_{\xi \xi} + 2u_{\xi \eta} + u_{\eta \eta}\\
    u_{tt} = c^2(u_{\xi \xi} - 2u_{\xi \eta} + u_{\eta \eta})  
\end{cases}\]
\[c^2(u_{\xi \xi} - 2u_{\xi \eta} + u_{\eta \eta}) = c^2(u_{\xi \xi} + 2u_{\xi \eta} + u_{\eta \eta}) \Longrightarrow 4u_{\xi \eta} = 0\]
\[u_{\xi \eta} = 0 \implies u_{\xi} = f(\xi) \implies u = F(\xi) + G(\eta)\]
\[u(x, t) = F(x - ct) + G(x + ct)\]

\textbf{Fourier Transform}
\[\F{u_{tt}} = c^2 \F{u_{xx}}\]
\begin{align*}
    \frac{d^2}{dt^2} \hat{u} &= c^2 (-i\kappa)^2 \hat{u}= -\kappa^2 c^2 \hat{u}\\
    \overset{y'' +ay=0}{\Longrightarrow} \hat{u} &= \hat{F}(\kappa)e^{i\kappa ct} + \hat{G}(\kappa)e^{-i\kappa ct}
\end{align*}
\[u = F(x - ct) + G(x + ct)\]
\end{document}